
\documentclass[ms.tex]{subfiles} 
\begin{document} 

\section{Conclusions} 
\label{sec:conclusions} 

\begin{itemize} 
	\item We have made use of the GCE models from~\citet{Johnson2021} which 
	characterize the Milky Way disc as a series of concentric rings with a 
	uniform~$\delta\rgal$ = 100 pc width. 
	These models treat each individual ring as a conventional one-zone model 
	of chemical evolution whil allowing individual stellar populations to move 
	between rings to include the effects of stellar migration. 

	\item We retain the IMF-averaged CCSN and SN Ia yields of O and Fe 
	from~\citet{Johnson2021}, adopt a zero yield of N from SN Ia, and 
	investigate theoretical and empirical CCSN N yields. 
	If our CCSN yield of O is accurate ($y_\text{O}^\text{CC} = 0.015$), then 
	the CCSN yield of N required to produce the ``plateau'' of [N/O] = $-0.7$ 
	at low [O/H] is~$y_\text{N}^\text{CC} = 3.6\times10^{-4}$. 
	Of the recent CCSN nucleosynthesis investigations, only~\citet{Limongi2018} 
	present yields for rotating progenitors. 
	With the non-rotating models falling short of this value, we find that 
	these yields from rotating massive stars are sufficient to explain the 
	N abundances seen at low [O/H], consistent with recent results 
	from~\citet{Grisoni2021}. 

	\item We find that there are substantial differences between the mass- and 
	metallicity-dependencies of the AGB star yields of N predicted from 
	theoretical models (see Fig.~\ref{fig:agb_yield_models}). 
	Although ascertaining the origins of the differences is difficult because 
	each model folds in different assumptions about, e.g., mass loss, 
	convection and convective boundaries, and nuclear reaction networks, the 
	differences can be qualitatively understood by the interaction between 
	TDU and HBB in massive AGB stars. 

	\item We find that in the absence of a starburst event, the gas phase 
	[N/O]-[O/H] relation is relatively time-independent up to lookback times 
	of~$\sim$6 - 8 Gyr in our fiducial model. 
	Similar to arguments that have been made regarding the low-[$\alpha$/Fe] 
	population in the Milky Way~\citep[e.g.][]{Schoenrich2009, Sharma2020, 
	Johnson2021}, we find that the [N/O]-[O/H] relation arises not out of an 
	evolutionary sequence but as a superposition of endpoints of multiple 
	evolutionary sequences. 
	That is, the time evolution of each ring through the [N/O]-[O/H] plane is 
	not the same line as the [N/O]-[O/H] relation that would be observed at the 
	present day in our model Galaxy (see Fig.~\ref{fig:no_oh_predictions}). 

	\item We find that the timescales for N production by a single stellar 
	population ($\sim$100 Myr) are considerably shorter than that of Fe 
	production in our models ($\sim$1 Gyr). 
	As a consequence, most N production occurs before a stellar population's 
	orbit will change significantly, and stellar migration has only a minimal 
	impact on the gas phase [N/O]-[O/H] relation. 
	Because the timescales for Fe production are longer,~\citet{Johnson2021} 
	find that the impact on the enrichment rates is significant, and may be 
	enough to explain the presence of young,~$\alpha$-rich stars in the solar 
	neighbourhood as seen in APOGEE~\citep{Chiappini2015, Martig2015, 
	Martig2016, Jofre2016, Yong2016, Izzard2018, SilvaAguirre2018, 
	Warfield2021}. 
	This difference underscores the argument from~\citet{Johnson2021} that in 
	order for nucleosynthetic yields to migrate along with their progenitor 
	stellar populations, the characteristic delay time for the production of 
	some element must be comparable to the timescales of stellar migration. 

	\item We find that no single AGB star yield model previously published in 
	the literature is able to reproduce the gas-phase [N/O]-[O/H] relation as 
	observed. 
	Our fiducial model with the~\cristallo~yields requires an artificial 
	amplification by a factor of~$\sim$3 in order to get the normalization 
	correct. 
	We find similar results with the~\ventura~yields (factor of~$\sim$2 
	amplification). 
	The~\karakasten~yields predict [N/O] to decrease with increasing [O/H]: a 
	slope with the wrong sign. 
	Although the~\karakas~yields predict a flat [N/O]-[O/H] relation, they're 
	able to explain the N abundances at low [O/H]; this is the only previously 
	published yield model in this paper that is able to explain the N 
	abundances at any metallicity. 
	Even with alternate forms for our CCSN yields as suggested by the 
	non-rotating models of~\citet{Woosley1995},~\citet{Nomoto2013}, 
	\citet{Sukhbold2016}, and~\citet{Limongi2018}, our model is able to 
	reproduce the observed [N/O]-[O/H] relation with neither 
	the~\karakasten~nor the~\karakas~yield models. 

	\item We find that our model is able to reproduce the correlations of [N/O] 
	with [O/Fe] and stellar age when N abundances are corrected for internal 
	mixing processes in stars. 
	\citet{Vincenzo2021} found that even after these corrections were made, the 
	N abundances of thin- and thick-disc stars persisted; our model is able to 
	reproduce this result. 
	\citet{Vincenzo2021} also found the [N/O]-age relation in bins of [Fe/H] to 
	be flat, whereas with uncorrected N abundances there is a significant 
	negative slope. 
	In agreement with~\citet{Vincenzo2021}, our model predicts the [N/O]-age 
	relation to be flat when a selection in [Fe/H] is made. 
	Both of these predictions trace back to our model predicting N abundances 
	to increase in the ISM at a similar rate as Fe out to lookback 
	times~$\lesssim$8 Gyr. 
	Because of this, a range in [Fe/H] maps approximately to a range in [N/O], 
	and as a result, [N/O] increases as [O/Fe] decreases. 

	\item We find that 25\% sinusoidal oscillations in the SFR and SFE induce 
	larger variability in the [N/O]-[O/H] relation than does stellar migration. 
	In detail, the fluctuations are mostly in [O/H] rather than in [N/O], and 
	this presents as a higher [N/O] at fixed [O/H], but our models suggest a 
	slightly more accurate statement is to characterize these fluctuations as 
	higher or lower [O/H] at fixed [N/O] (see discussion 
	in~\S~\ref{sec:results:schaefer_comp}). 
	We find that these models can accurately explain the intrinsic scatter in 
	the gas phase [N/O]-[O/H] relation as inferred observationally 
	by~\citet{Schaefer2020}, consistent with their argument and supporting 
	their conjecture that radial migration is not the dominant source of 
	scatter in their data. 
	Although real galaxies likely have neither perfectly sinusoidal variations 
	in their SFR or SFE nor with a constant amplitude, these models demonstrate 
	clearly that such variability induces fluctuations in the gas phase N and O 
	abundances larger than that caused by stellar migration. 

	\item The results outlined in this paper highlight the importance of 
	empirically calibrated yields of all elements from all nucleosynthetic 
	sources in GCE models. 
	The combination of theoretically predicted yields with flexible 
	computational tools such as~\vice~can provide powerful constraints for 
	future models of stellar evolution and element production. 
\end{itemize} 

\end{document} 

