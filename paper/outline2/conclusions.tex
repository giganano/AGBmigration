
\documentclass[ms.tex]{subfiles} 
\begin{document} 

\section{Conclusions} 
\label{sec:conclusions} 

\begin{itemize} 
	\item We have made use of the GCE models from~\citet{Johnson2021} which 
	characterize the Milky Way disc as a series of concentric rings with a 
	uniform~$\delta\rgal$ = 100 pc width. 
	These models treat each individual ring as a conventional one-zone model 
	of chemical evolution while allowing individual stellar populations to move 
	between rings to include the effects of stellar migration. 

	\item We retain the IMF-averaged CCSN and SN Ia yields of O and Fe 
	from~\citet{Johnson2021}, adopt a zero yield of N from SN Ia, and 
	investigate theoretical and empirical CCSN N yields. 
	If our CCSN yield of O is accurate ($y_\text{O}^\text{CC} = 0.015$), then 
	the CCSN yield of N required to produce the ``plateau'' of [N/O] = $-0.7$ 
	at low [O/H] is~$y_\text{N}^\text{CC} = 3.6\times10^{-4}$. 
	Of the recent CCSN nucleosynthesis investigations, only~\citet{Limongi2018} 
	present yields for rotating progenitors. 
	With the non-rotating models falling short of this value, we find that the 
	effects of rotation are necessary to explain the N abundances seen at low 
	[O/H], consistent with recent results from~\citet{Grisoni2021}. 

	\item There are substantial differences between the mass- and 
	metallicity-dependencies of the AGB star yields of N predicted from 
	theoretical models (see Fig.~\ref{fig:agb_yield_models}). 
	Although ascertaining the origins of the differences is difficult because 
	each model folds in different assumptions about, e.g., mass loss, 
	convection and convective boundaries, and nuclear reaction networks, the 
	differences can be qualitatively understood by the interaction between 
	TDU and HBB in massive AGB stars (see discussion 
	in~\S~\ref{sec:yields:agb}). 
	{\color{red} 
	Potentially give a little synopsis here in addition to 
	the~\S~\ref{sec:yields:agb} reference. 
	} 

	\item The IMF-weighted~\cristallo~yields are relatively independent of 
	progenitor mass, while the other models considered here have more 
	substantial yields from higher mass AGB stars. 
	In all cases, the characteristic delay times for production are short 
	compared to the SN Ia timescale ($\sim$few hundred Myr compared to~$\sim$1 
	Gyr). 
	{\color{red} 
	(This part is something of a non-sequitur, but I think it's worth saying 
	somewhere; perhaps there's a better place for it.) 
	As a general result, we caution against interpretations of AGB star 
	nucleosynthesis which attribute a single DTD or characteristic delay-time 
	to it. 
	The time-dependence of AGB star production is entirely dictated by how the 
	yield depends on progenitor mass and the adopted mass-lifetime relation. 
	Because different elements are produced in different amounts by AGB stars 
	of different masses, the DTD is generally different from element to 
	element (compare, e.g., Fig.~\ref{fig:ssp} here for N to Fig. 5 
	of~\citealt{Johnson2020} for strontium). 
	Interpretations that place a single characteristic delay-time on this 
	enrichment process could lead to potentially spurrious conclusions. 
	}

	\item For a smooth SFH, the gas phase [N/O]-[O/H] relation is relatively 
	time-independent up to lookback times of~$\sim$6 - 8 Gyr in our fiducial 
	model. 
	Similar to arguments that have been made regarding the low-[$\alpha$/Fe] 
	population in the Milky Way~\citep[e.g.][]{Schoenrich2009, Sharma2020, 
	Johnson2021}, we find that the [N/O]-[O/H] relation arises not out of an 
	evolutionary sequence but as a superposition of endpoints of multiple 
	evolutionary sequences. 
	That is, the time evolution of each ring through the [N/O]-[O/H] plane is 
	not the same line as the [N/O]-[O/H] relation that would be observed at the 
	present day in our model Galaxy (see Fig.~\ref{fig:no_oh_predictions}). 

	\item Our models suggest that stellar migration has only a small impact on 
	the N enrichment rate at a given radius and time because the characteristic 
	delay-times of N production are significantly shorter than the migration 
	timescale. 
	In the case of Fe, the characteristic delay times are longer ($\sim$1 Gyr 
	compared to a~$\sim$few hundred Myr), and based on this~\citet{Johnson2021} 
	find that the impact of migration on Fe enrichment rates is considerable 
	(as much as a factor of~$\sim$3 in the SN Ia rate). 
	The effect is sufficiently strong such that it may be enough to explain the 
	intrinsically young sub-component of the young~$\alpha$-rich stars observed 
	in the solar neighbourhood~\citep{Chiappini2015, Martig2015, Martig2016, 
	Jofre2016, Yong2016, Izzard2018, SilvaAguirre2018, Warfield2021}. 
	Although our model suggests the impact may be larger when the Galaxy was 
	young, this effect is only at the~$\lesssim0.05$ dex level for N. 
	This difference underscores the argument from~\citet{Johnson2021} that in 
	order for nucleosynthetic yields to migrate along with their progenitor 
	stellar populations, the characteristic delay time for the production of 
	some element must be comparable to the timescales of stellar migration. 

	\item We are unable to reproduce the gas phase [N/O]-[O/H] relation 
	as observed with any of the AGB star yield tables investigated here. 
	The~\cristallo~yields require an artificial amplification by a factor 
	of~$\sim$3 in order to get the normalization correct, and we find similar 
	results with the~\ventura~yields (factor of~$\sim$2 amplification). 
	The~\karakasten~yields predict [N/O] to decrease with increasing [O/H]: a 
	slope with the wrong sign. 
	Although the~\karakas~yields predict a relatively flat [N/O]-[O/H] 
	relation, they're able to explain the N abundances at low [O/H]; this is 
	the only previously published yield model in this paper that is able to 
	explain the N abundances at any metallicity without modification. 
	Even with alternate forms for our CCSN yields as suggested by the 
	non-rotating models of~\citet{Woosley1995},~\citet{Nomoto2013}, 
	\citet{Sukhbold2016}, and~\citet{Limongi2018}, our model is able to 
	reproduce the observed [N/O]-[O/H] relation with neither 
	the~\karakasten~nor the~\karakas~yield models. 
	{\color{red} 
	There may be additional things to say on this if our investigation with 
	a lowered O yield and a lowered~$\eta$ turns up something interesting. 
	Fiorenzo was also able to do this with a differential wind. 
	}

	\item Our model successfully reproduces the correlations of [N/O] 
	with [O/Fe] and stellar age when N abundances are corrected for internal 
	mixing processes in stars. 
	This is an important success of the model, because~\citet{Vincenzo2021} 
	found that even after these corrections were made, the dichotomy in N 
	abundances of thin- and thick-disc stars persisted. 
	\citet{Vincenzo2021} also found the [N/O]-age relation in bins of [Fe/H] to 
	be flat, whereas with uncorrected N abundances there is a significant 
	negative slope. 
	In agreement with~\citet{Vincenzo2021}, our model predicts the [N/O]-age 
	relation to be flat when a selection in [Fe/H] is made. 
	Both of these predictions trace back to our model predicting N abundances 
	to increase in the ISM to correlate with the Fe abundance out to lookback 
	times of~$\sim$10 Gyr; this correlation with Fe is significantly stronger 
	than its correlation with O. 
	Because of this, a range in [Fe/H] maps approximately to a range in [N/O], 
	and [N/O] increases as [O/Fe] decreases, maintaining the chemical 
	dichotomy between the two sequences. 
	Even though this model is unsuccessful at reproducing the bimodality in 
	[$\alpha$/Fe] in detail~\citep{Johnson2021}, a future version of the model 
	which does will still predict such a dichotomy in N as well due to this 
	correlation. 

	\item We find that 25\% sinusoidal oscillations in the SFR and SFE induce 
	larger variability in the [N/O]-[O/H] relation than does stellar migration. 
	Although this can be interpreted as higher/lower [N/O] at fixed [O/H], our 
	models suggest that it's more accurate to characterize the scatter as a 
	higher/lower [O/H] at fixed [N/O] (see discussion 
	in~\S~\ref{sec:results:schaefer_comp}). 
	With 25\% variability in the SFE, we find that our model can accurately 
	explain the intrinsic scatter in the gas phase [N/O]-[O/H] relation as 
	inferred observationally by~\citet{Schaefer2020}. 
	This supports their conjectures that there is a causal relationship behind 
	this correlation and that radial migration is not a significant 
	contributing factor to the scatter in their data. 
	Although real galaxies likely have variations in their SFR or SFE that are 
	neither perfectly sinusoidal nor with a constant amplitude, such effects 
	induce variations in the [N/O]-[O/H] plane which, in a sufficiently large 
	sample, would present observationally as intrinsic scatter.  
	Our models suggest that typical variations in these 
	quantities cause variability in the relation which is considerably larger 
	than that caused by stellar migration. 

	\item The results outlined in this paper highlight the importance of 
	empirically calibrated yields of all elements from all nucleosynthetic 
	sources in GCE models. 
	The combination of theoretically predicted yields with flexible 
	computational tools such as~\vice~can provide powerful constraints for 
	future models of stellar evolution and element production. 
\end{itemize} 

\end{document} 

