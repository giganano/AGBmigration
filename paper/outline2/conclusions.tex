
\documentclass[ms.tex]{subfiles} 
\begin{document} 

\section{Conclusions} 
\label{sec:conclusions} 

\begin{itemize} 
	\item We have made use of the GCE models from~\citet{Johnson2021} which 
	characterize the Milky Way disc as a series of concentric rings with a 
	uniform~$\delta\rgal$ = 100 pc width. 
	These models treat each individual ring as a conventional one-zone model 
	of chemical evolution while allowing individual stellar populations to move 
	between rings to include the effects of stellar migration. 

	\item We retain the IMF-averaged CCSN and SN Ia yields of O and Fe 
	from~\citet{Johnson2021}, adopt a zero yield of N from SN Ia, and 
	investigate theoretical and empirical CCSN N yields. 
	If our CCSN yield of O is accurate ($y_\text{O}^\text{CC} = 0.015$), then 
	the CCSN yield of N required to produce the ``plateau'' of [N/O] = $-0.7$ 
	at low [O/H] is~$y_\text{N}^\text{CC} = 3.6\times10^{-4}$. 
	Of the recent CCSN nucleosynthesis investigations, only~\citet{Limongi2018} 
	present yields for rotating progenitors. 
	With the non-rotating models falling short of this value, we find that the 
	effects of rotation are necessary to explain the N abundances seen at low 
	[O/H], consistent with recent results from~\citet{Grisoni2021}. 

	\item There are substantial differences between the mass- and
	metallicity-dependencies of the AGB star yields of N predicted from
	theoretical models (see Fig.~\ref{fig:agb_yield_models}).
	Ascertaining the origin of these differences is diffuclt because each
	model folds in different assumptions about, e.g., mass loss, opacities of
	various isotopic species, convection and convective boundaries, and nuclear
	reaction networks.
	Nonetheless the differences can be qualitatively understood by considering
	the differences in how HBB and TDU proceed in the stellar models (see
	discussion in~\S~\ref{sec:yields:agb}).

	\item When weighted by the IMF, the~\cristallo~yields have the smallest
	contribution from the highest mass AGB stars.
	Each additional model shows a considerable contribution from AGB stars
	with ZAMS masses of~$\gtrsim 3 M_\odot$.
	In all cases, the characteristic delay times for production are short
	compared to the SN Ia timescale ($\sim$few hundred Myr compared  to~$\sim$1
	Gyr).
	This result, however, likely does not extend to other elements produced in
	AGB stars because having yields from these higher mass AGB stars makes N
	somewhat unique (see discussion in~\S~\ref{sec:results:yields}).

	\item For a smooth SFH, the gas phase [N/O]-[O/H] relation is relatively
	time-independent up to lookback times of~$\sim$6 - 8 Gyr in our fiducial
	model.
	Similar to arguments that have been made regarding the low-[$\alpha$/Fe]
	population in the Milky Way~\citep[e.g.][]{Schoenrich2009, Sharma2020,
	Johnson2021}, we find that the [N/O]-[O/H] relation arises not out of an
	evolutionary sequence but as a superposition of endpoints of multiple
	evolutionary sequences.
	That is, the time evolution of each Galactic region through the [N/O]-[O/H]
	plane is not the same line as the [N/O]-[O/H] relation that would be
	observed at the present day in our model Galaxy (see
	Fig.~\ref{fig:no_oh_predictions}).

	\item We are unable to reproduce the gas phase [N/O]-[O/H] relation in our
	fiducial model with any of the AGB star yield tables investigated here.
	The~\cristallo~yields require an artificial amplification by a factor of
	$\sim$3 in order to get the normalization correct, and we find similar
	results with the~\ventura~yields (factor of~$\sim$2 amplification).
	We can also mitigate this difference by decreasing our CCSN yields of O
	and our mass-loading factor~$\eta$ as a function of Galactic radius by
	similar factors.
	This is physically plausible if a substantial fraction of high mass stars
	collapse to black holes instead of ending their lives in CCSNe.
	This suggests that either N yields from AGB stars are a factor of~$2 - 3$
	times larger than stellar evolution models suggest or that failed supernovae
	are quite common; it could also be a mix of the two effects.
	With the~\karakasten~yields, our model predicts [N/O] to decrease with
	increasing [O/H] - a slope with the wrong sign.
	The~\karakas~yields predict a relatively flat [N/O]-[O/H] relation due to
	updates to these models which increase the N yields at high [O/H] but
	decrease the N yields at low [O/H] (see discussion
	in~\S~\ref{sec:yields:agb}).
	Even with alternate forms for our CCSN yields as suggested by the
	non-rotating models of~\citet{Woosley1995},~\citet{Nomoto2013},
	\citet{Sukhbold2016}, and~\citet{Limongi2018}, we are unable to
	reproduce the observed [N/O]-[O/H] relation with either the~\karakasten~or
	the~\karakas~AGB star yields.
	In general, in order to reproduce this relation with the correct slope,
	we find that the total N yield from both CCSNe and AGB stars must increase
	roughly linearly with metallicity (see, e.g., the~\cristallo,~\ventura, and
	linear models in the right panel of Fig.~\ref{fig:ssp}), but the
	normalization of the N yields required to get the [N/O]-[O/H] relation
	correct depends on the normalization of the O yield.

	\item To test our models against N abundances observed in stars, we use the
	measurements from~\citet{Vincenzo2021}.
	They use~\texttt{MESA} stellar evolution models to correct the
	spectroscopically derived measurements for internal mixing processes (i.e.
	dredge-up dilution) known to affect the photospheric N abundances in
	evolved stars.
	They find that [N/O] in bins of [Fe/H] is roughly independent of stellar age
	up to age of~$\sim$10 Gyr, a result our model successfully reproduces.
	This is a notable success of our model and the~\citet{Vincenzo2021}
	measurements because with uncorrected N abundances, [N/O] at fixed [Fe/H]
	does depend on age.
	Our model predicts [N/O] to correlate inversly with [O/Fe], a result which
	\citet{Vincenzo2021} derive for both corrected and uncorrected N abundances.
	This suggests that even when N abundances are corrected for internal mixing
	processes, the chemical dichotomy between the thin and thick discs of the
	Galaxy persists.
	Both of these results trace back to a correlation between N and Fe
	abundances in the ISM predicted by the model.
	Although the characteristic N production timescale from single stellar
	populations is short ($\sim$200 Myr), metallicity dependent yields dictate
	that more abundance species like O and Fe must be produced in substantial
	amounts before N yields from AGB stars can become significant.
	As a consequence, N production by many stellar populations is more
	delayed than one might expect given the characteristic delay time from a
	single stellar population.
	This results in [N/H] exhibiting a stronger correlation with [Fe/H] than
	[O/H] up to lookback times of~$\sim$10 Gyr in the gas phase (see
	Fig.~\ref{fig:xh_lookback}).
	Because of this, [N/O] increases and [O/Fe] decreases at fixed [O/H], and
	the chemical dichotomy between the two disc populations arises.
	Furthermore, with [O/H] roughly flat up to lookback times of~$\sim$10 Gyr
	due to its quick production timescale, a bin in [Fe/H] corresponds roughly
	to a bin in [N/O], which explains the flat [N/O]-age relation at fixed
	[Fe/H].
	Although this model is unsuccessful at reproducing the bimodality in
	[$\alpha$/Fe] in detail~\citep{Johnson2021}, a future version of the model
	which does will still predict chemical differences in N due to this
	correlation.

	\item To investigate the sources of intrinsic scatter in the [N/O]-[O/H]
	relation, we construct two variations of the fiducial model from
	\citet{Johnson2021}.
	In these alternate scenarios, we impose 25\% sinusoidal oscillations on one
	of the SFE or the SFR as functions of time while incorporating the effects
	of stellar migration on the enrichment rates in both cases.
	These oscillations are characteristic of what we see in~\hsim, the
	hydrodynamical simulation from which our model's dynamical history is
	drawn.
	{\color{red} (Observational refs)}
	In general, we find that these 25\% oscillations induce much larger
	variability in the gas-phase N and O abundances than does stellar
	migration.
	This is because of the quick production timescales for N ($\sim$few hundred
	Myr, see Fig.~\ref{fig:ssp}) - there simply isn't much time for orbits to
	dynamically evolve before most of a stellar population's N is ejected to
	the ISM.
	This is the opposite of what~\citet{Johnson2021} find for Fe.
	Because SN Ia enrichment has a characteristic delay time closer to~$\sim$1
	Gyr and a significant tail to long delay times, a substantial fraction of
	Fe production happens on timescales similar to the migration timescale in
	this model.
	Consequently, the impact of migration on Fe enrichment rates is sometimes 
	as high as a factor of~$\sim$3 (see discussion in their~\S\S~3.1 and 3.4).
	This is sufficiently strong such that it can explain the intrinsically
	young sub-component of the young~$\alpha$-rich stars observed in the solar
	neighbourhood~\citep{Chiappini2015, Martig2015, Martig2016, Jofre2016,
	Yong2016, Izzard2018, SilvaAguirre2018, Warfield2021}.
	Although our model suggests that the impact may be larger when the Galaxy
	was young, this effect is only at the~$\lesssim$0.5 dex level for N.
	This difference underscors the argument from~\citet{Johnson2021} that in
	order for nucleosynthetic yields to migrate along with their progenitor
	stellar populations, the characteristic delay time for the production of
	some element must be comparable to the timescales of stellar migration.

	\item We compare our models with 25\% oscillations in the SFE and the SFR
	to the observational measurements from~\citet{Schaefer2020}.
	They demonstrate using data from the MaNGA IFU survey that the scatter in
	the [N/O]-[O/H] relation at fixed galaxy mass is correlated with variations
	in the local SFE, with lower SFE systems exhibiting higher [N/O] at fixed
	[O/H].
	Although this is predicted by simple one-zone chemical evolution models
	\citep[e.g.][]{Molla20016, Vincenzo2016a}, they were unable to rule out
	stellar migration as another potential source of scatter in the relation.
	Our oscillatory models with this 25\% amplitude span a range in [N/O] at
	fixed [O/H] which is in good agreement with the width of the [N/O]-[O/H]
	relation that~\citet{Schaefer2020} derive for Milky Way mass galaxies
	($10^{10.5} - 10^{11} M_\odot$).
	Although stellar migration has a smaller effect on the enrichment rates, it
	is not a negligible source of scatter compared to the~\citet{Schaefer2020}
	distributions (see Fig.~\ref{fig:schaefer_comp}) but is still sub-dominant
	compared to variability in either the SFE or the SFR.
	Although real galaxies likely have variations in their SFR of SFE that are
	neither perfectly sinusoidal nor with constant amplitude, such effects
	induce variations in the [N/O]-[O/H] plane which should be captured by our
	models, and in a sufficiently large sample, this would present
	observationally as intrinsic scatter.

	\item The results outlined in this paper highlight the importance of 
	empirically calibrated yields of all elements from all nucleosynthetic 
	sources in GCE models. 
	The combination of theoretically predicted yields with flexible 
	computational tools such as~\vice~can provide powerful constraints for 
	future models of stellar evolution and element production. 
\end{itemize} 

\end{document} 

