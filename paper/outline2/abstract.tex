
\documentclass[ms.tex]{subfiles}
\begin{document}

\begin{abstract}
We use a multi-ring galactic chemical evolution model to probe the
astrophysical production of nitrogen (N) in the Milky Way.
This approach treats individual annuli in the Galaxy disc as conventional
one-zone models, and to include the effects of radial migration, stellar
populations move between annuli in a manner based on star particles from a
hydrodynamical simulation.
We find that some recent AGB star yield tables are able to reproduce the
gas-phase [N/O]-[O/H] relation as observed only if a substantial fraction of
massive stars collapse to black holes.
If instead most massive stars explode as supernovae, we must artificially
increase N yields from AGB stars by factors of~$2 - 3$ to offset the
additional oxygen.
We demonstrate that, with a viable set of AGB star yields, our model is able to
reproduce many of the observed correlations between N, O, and Fe abundances for
stars when the N abundances are corrected for internal mixing processes within
stars.
With any of these yields, N production timescales are sufficiently short such
that stellar migration is only a minimal source of intrinsic scatter in the
observed [N/O]-[O/H] relation.
Modest variations in the star formation rate and star formation efficiency
produce considerably larger variations in the gas phase N and O abundances,
consistent with previous observational arguments.
Our models run using the publicly available~\texttt{Versatile Integrator for
Chemical Evolution} (\texttt{VICE};~\url{https://pypi.org/project/vice}).
{\color{red} To do: Play around with models with no metallicity dependence
and no time dependence on the AGB N yields.
}
\end{abstract}

\end{document}

