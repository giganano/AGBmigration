
\documentclass[ms.tex]{subfiles} 
\begin{document} 

\section{Introduction} 
\label{sec:intro} 
\begin{itemize} 
	\item In terms of astrophysical nucleosynthesis, nitrogen (N) is a unique 
	element. 
	\begin{itemize} 
		\item It's one of only a few elements lighter than strontium (Z = 38) 
		with significant nucleosynthetic yields from asymptotic giant branch 
		(AGB) stars~\citep{Johnson2019}. 

		\item Alongside helium, it is one of the primary nuclear fusion 
		products of main sequence stars more massive than the sun with nonzero 
		metallicity. 
		The CNO cycle catalyses the proton-proton chain of nuclear 
		reactions~\citep*[e.g.][]{Suliga2020} using carbon (C), N, and oxygen 
		(O) target nuclei,\footnote{
			\Ctwelve(p,$\gamma$)\Nthirteen($\beta^+$,$\nu_e$)\Cthirteen 
			(p,$\gamma$)\Nfourteen(p,$\gamma$)\Ofifteen
			($\beta^+$,$\nu_e$)\Nfifteen(p,$\alpha$)\Ctwelve 
		} the slowest component of which is 
		the~\Nfourteen(p,$\gamma$)\Ofifteen~reaction. 
		This bottleneck is strong enough that to first order, the effect of the 
		CNO cycle is to convert all C and O isotopes in a star into~\Nfourteen. 

		\item It's among a select group of elements whose observed abundances 
		in stellar spectra often do not reflect the star's birth abundances. 
		The effect of internal mixing processes changes the atmospheric 
		composition of red giants, a phenomenon both expected from theoretical 
		models and observed in open and globular clusters~\citep{Gilroy1989, 
		Korn2007, Lind2008, Souto2018, Souto2019}. 
	\end{itemize} 

\end{itemize}

\begin{figure*} 
\centering 
\includegraphics[scale = 0.5]{no_oh_observed.pdf} 
\caption{ 
The [N/O]-[O/H] relation observed in HII regions in nearby NGC spiral galaxies 
(grey X's:~\citealp*{Pilyugin2010}), in HII regions in blue, diffuse star 
forming dwarf galaxies (red triangles:~\citealp{Berg2012}; 
green stars:~\citealp*{Izotov2012}; blue diamonds:~\citealp{James2015}), in 
local stars and HII regions (purple circles:~\citealp{Dopita2016}), and in the 
MaNGA IFU survey (black squares:~\citealp{Belfiore2017}). 
The fit to [N/O] as a function of [O/H] in Galactic and extragalactic HII 
regions by~\citet*{Henry2000} is shown in a black dotted line. 
The Sun, at (0, 0) on this plot by definition, is marked by a red star. 
We omit the uncertainties for visual clarity. 
{\color{red} To do: Pull the data for CHAOS galaxies from Danielle Berg's 
and Noah Rogers's papers and add them to this figure. 
} 
} 
\label{fig:no_oh_observed} 
\end{figure*} 

\begin{itemize} 
	\item Both observationally and theoretically, N is among the more well 
	studied elements. 
	Fig.~\ref{fig:no_oh_observed} presents a compilation of observed abundances 
	of N and O in the gas phase: 
	\begin{itemize} 
		\item HII regions in nearby NGC spirals~\citep{Pilyugin2010} 

		\item HII regions in blue, diffuse star forming dwarf 
		galaxies~\citep{Berg2012, Izotov2012, James2015} 

		\item Local stars and HII regions~\citep{Dopita2016} 

		\item Galactic and extragalactic HII regions~\citep{Henry2000} 

		\item Star-forming regions in 550 nearby galaxies in the MaNGA IFU 
		survey~\citep{Belfiore2017} 
	\end{itemize} 
	Despite intrinsic scatter and some systematic variation in how the 
	abundances are determined, the [N/O]-[O/H]\footnote{
		We follow the standard notation where 
		[X/Y]~$\equiv \log_{10}(X/Y) - \log_{10}(X/Y)_\odot$. 
	} relation is more or less the same across a wide range of physical 
	enrivonments. 

	\item In this paper, we're interested in the origin of both the shape and 
	scatter in this trend. 

	\item N is not unique in that perhaps the largest source of uncertainty in 
	modeling its abundances is that accurate and precise nucleosynthetic yields 
	remain elusive. 
	Presently, no combination of core collapse supernova explosion model and 
	black hole landscape has been able to reproduce the observed abundance 
	pattern of the elements, and nitrogen is no exception~\citep{Griffith2021}. 
	Furthermore, theoretical models of AGB stars predict different N yields as 
	a function of progenitor mass and metallicity (see discussion in, 
	e.g.~\S~5 of~\citealp{Karakas2016} comparing their models to that 
	of~\citealp{Cristallo2011, Cristallo2015}). 

	\item In this paper, we aim to constrain N yields from AGB stars 
	empirically by assessing their ability to reproduce the observed abundance 
	correlations between N and O, such as that illustrated in 
	Fig.~\ref{fig:no_oh_observed}. 
	\citet{Vincenzo2021} demonstrate that when N abundances are corrected for 
	internal mixing processes, the correlations with stellar age and other 
	elemental abundances are affected; whether or not these estimates of the 
	birth abundances can be reproduced in galactic chemical evolution (GCE) 
	models is also of central interest to this paper. 

	\item In a sample of 6,507 galaxies from the Mapping Galaxies at Apache 
	Point Observatory survey~\citep[MaNGA;][]{Bundy2015},~\citet{Schaefer2020} 
	recently argued that intrinsic scatter in the [N/O]-[O/H] relation is a 
	consequence of variations in the local star formation efficiency. 
	In regions of slower star formation, the [N/O] ratio tends to be slightly 
	higher at fixed [O/H] (see their Fig. 4), as expected from GCE 
	models~\citep[e.g.][]{Molla2006, Vincenzo2016a}. 
	However,~\citet{Schaefer2020} could not rule out radial migration as an 
	additional source of scatter in the gas phase [N/O]-[O/H] relation. 
	Investigating GCE models for O and iron (Fe) abundances in the Milky Way 
	which track the positions of stars as they migrate within the 
	disc,~\citet{Johnson2021} found that the characteristic delay time of type 
	Ia supernovae (SNe Ia) is sufficiently long such that stellar migration 
	has a noticeable impact on the Fe abundance in the ISM at a given 
	Galactocentric radius and time. 
	Since N is produced in significant quantities by AGB stars, which like SNe 
	Ia are delayed nucleosynthetic sources, it's possible that its gas phase 
	abundance could be affected by a deficit or surplus of AGB stars induced 
	by radial migration; in a sufficiently large sample of galaxies like MaNGA 
	this would present observationally as scatter in the gas phase abundances. 
	The question of whether one of radial migration or star formation 
	efficiency dominates over the other in driving this scatter is also of 
	central interest in this paper. 
\end{itemize} 

\end{document} 

