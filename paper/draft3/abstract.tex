
\documentclass[ms.tex]{subfiles}
\begin{document}

\begin{abstract}
We derive empirical constraints on the nucleosynthetic yields of nitrogen by
incorporating N enrichment into our previously developed and empirically tuned
multi-zone galactic chemical evolution model.
We adopt a metallicity-independent (``primary'') N yield from massive stars and
a metallicity-dependent (``secondary'') N yield from AGB stars.
In our model, galactic radial zones do not evolve along the observed
[N/O]-[O/H] relation, but first increase in [O/H] at roughly constant [N/O],
then move upward in [N/O] via secondary N production.
By~$t\approx5$ Gyr, the model approaches an equilibrium [N/O]-[O/H] relation,
which traces the radial oxygen gradient.
{\color{red}
Reproducing the [N/O]-[O/H] trend observed in extra-galactic systems constrains
the ratio of IMF-averaged N yields to the IMF-averaged O yield of core collapse
supernovae.  We find good agreement if we adopt
$\ycc{N}/\ycc{O}=0.024$ and $y_\text{N}^\text{AGB}/\ycc{O} = 0.062(Z/Z_\odot)$.
For the theoretical AGB yields we consider, simple stellar populations 
release half their N after only $\sim$250 Myr.
Our model reproduces the [N/O]-[O/H] relation found for Milky Way stars
in the APOGEE survey, and it reproduces (though imperfectly) the trends of
stellar [N/O] with age and [O/Fe].
The metallicity-dependent yield plays the dominant role in shaping the gas-phase
[N/O]-[O/H] relation, but the AGB time-delay is required to match the stellar
age and [O/Fe] trends.
If we add~$\sim$40\% oscillations to the star formation rate, the model
reproduces the scatter in the gas phase [N/O]-[O/H] relation observed 
in external galaxies by MaNGA.
We discuss implications of our results for theoretical models of N production
by massive stars and AGB stars.
}
\end{abstract}

\end{document}

