
\documentclass[ms.tex]{subfiles}
\begin{document}

\section{The Multi-Zone Chemical Evolution Model}
\label{sec:multizone}

We use the fiducial model for the Milky Way published by
\citet{Johnson2021}, which runs using the~\vice~GCE code (see Appendix
\ref{sec:vice};~\citealp{Johnson2020, Griffith2021}).
% \citet{Johnson2021} focus their discussion on the predicted O and Fe
% abundances, but because~\vice~recognizes most elements on the periodic table,
% computing N abundances with this model is easy.
Multi-zone models allow simultaneous calculations of abundances for multiple
Galactic regions, making them a more physically realistic option than classical
one-zone models for a system like the Milky Way.
Furthermore, they can take into account stellar migration in a framework that
is much less computationally expensive than hydrodynamical simulations, making
them the ideal experiments for our purposes.
We provide a brief summary of the model here, but a full breakdown can be found
in~\S~2 of~\citet{Johnson2021}.
\par
As in previous models for the Milky Way with similar motivations
\citep[e.g.][]{Matteucci1989, Schoenrich2009, Minchev2013, Minchev2014,
Minchev2017, Sharma2021}, this model parametrizes the Galaxy disc as a series
of concentric rings.
With a uniform width of~$\delta\rgal = 100$ pc, each ring is assigned its own
{\color{red} \sout{star formation history (SFH)} SFH}, and with assumptions
about outflows and the
$\Sigma_\text{gas}-\dot{\Sigma}_\star$ relation (see discussion
below),~\vice~computes the implied amounts of gas and infall at each timestep
automatically.
Under the caveat that stellar populations can move and place some of their
nucleosynthetic products in rings other than the one they were born in, each
ring is otherwise described by a conventional one-zone GCE model.
Allowing stars to enrich distributions of radii was a novel addition to this
type of model, and~\citet{Johnson2021} demonstrated that this has a
significant impact on enrichment rates from delayed sources such as SNe Ia.
\par
To drive stellar migration, the model makes use of star particles from a
hydrodynamical simulation, for which~\citet{Johnson2021} select the~\hsim~galaxy
from the~\citet{Christensen2012} suite evolved with the N-body+SPH code
\texttt{GASOLINE}~\citep{Wadsley2004}; we retain this decision here.
\hsim~spans 13.7 Gyr of evolution, but the sample of star particles with
reliable birth radii span 13.2 Gyr in age; the model thus places the onset of
star formation~$\sim$500 Myr after the Big Bang and integrates up to the
present day.
Previous studies have shown that~\hsim, among other disc galaxies evolved with
similar physics, has a realistic rotation curve~\citep{Governato2012,
Christensen2014a, Christensen2014b}, stellar mass~\citep{Munshi2013},
metallicity~\citep{Christensen2016}, dwarf satellite population
\citep{Zolotov2012, Brooks2014}, HI properties~\citep{Brooks2017}, and stellar
age-velocity relation~\citep{Bird2021}.
Despite this, there are some interesting differences between~\hsim~and the
Milky Way.
First and foremost,~\hsim~had only a weak and transient bar and lacks one at
the present day, while the Milky Way is known to have a strong, long-lived
central bar~\citep[e.g.][]{Bovy2019}.
This could indicate that the dynamical history of~\hsim~and its star particles
differs significantly from that of the Milky Way.
Furthermore, the last major merger in~\hsim~was at a redshift of~$z \approx 3$,
making it an interesting case study for its quiescent merger history
\citep[e.g.][]{Zolotov2012}, while the Sagitarrius dwarf galaxy is believed to
have made pericentric passages around the Milky Way at~$1 - 2$ Gyr
intervals~\citep{Law2010}.
% With this in mind, a dynamical history such as that of~\hsim~in this GCE model
% can be thought of as capturing an evolutionary history much more secular than
% the Milky Way.
Although these differences between~\hsim~and the Milky Way are well understood,
their impact on chemical evolution is not.
We are unaware of any studies that investigate the impact of different
assumptions regarding the Galaxy's dynamical history on predicted abundances;
this is however an interesting question for future work.
\par
Radial migration of stars proceeds from the~\hsim~star particles in a simple
manner; for a stellar population in our model born at a radius~$R_\text{birth}$
and a time~$t_\text{birth}$,~\vice~searches for star particles born at
$R_\text{birth} \pm 250$ pc and $t_\text{birth} \pm 250$ Myr.
From the star particles that make this cut, it then randomly selects one to act
as that stellar population's~\textit{analogue}.
The stellar population then assumes the present day midplane distance~$z$ and
the change in orbital radius~$\Delta\rgal$ of its analogue between its birth
and the present day.
In the~\citet{Johnson2021} fiducial model, stellar populations move to their
implied final radii with a~$\sqrt{\text{age}}$ dependence according to:
\begin{equation}
\rgal(T) = R_\text{birth} + \Delta\rgal
\sqrt{\frac{t - t_\text{birth}}{13.2~\text{Gyr} - t_\text{birth}}},
\end{equation}
where 13.2 Gyr is simply the present day (see discussion above).
With displacement proportional to~$\sqrt{\text{age}}$, this corresponds to a
scenario in which radial migration proceeds as a diffusion process as modeled
by~\citet{Frankel2018, Frankel2019} and supported by the N-body simulations
of~\citet*{Brunetti2011}.
Although~\citet{Johnson2021} investigated other assumptions for this
time-dependence, in the present paper we only use this parametrization
(hereafter referred to as the ``diffusion'' prescription) and an idealized one
in which stars remain at their birth radius until they instantaneously migrate
at the present day (hereafter referred to as the ``post-processing''
prescription).
If~\vice~does not find any star particles from~\hsim~in its initial
$\rgal \pm 250$ pc and~$t \pm 250$ Myr search, it widens it to
$\rgal \pm 500$ pc and~$t \pm 500$ Myr; if still no candidate analogues are
found,~\vice~maintains the~$t \pm 500$ Myr requirement, but assigns the star
particle with the smallest difference in birth radius as the analogue.
This procedure can be thought of as ``injecting'' the dynamics of
the~\hsim~galaxy into our multi-zone chemical evolution model, and it can in
principle be repeated for any hydrodynamical simulation of a disc galaxy.
As in~\citet{Johnson2021}, we neglect radial gas flows
\citep[e.g.][]{Lacey1985, Bilitewski2012, Vincenzo2020}, instead focusing on
the impact of stellar migration.
\par
Rather than using a hydrodynamical simulation, some previous studies have
implemented stellar migration using dynamical arguments
\citep[e.g.][]{Schoenrich2009, Sharma2021}.
An advantage of our approach over this is that these dynamical arguments
introduce free parameters which then require fitting to data.
It is also unclear to what extent the fit might bias the model into agreement
with quantities in the sample not involved in the fit.
A disadvantage of our approach is that we are restricted to one realization of
our dynamical history; slight variations are not possible.
We do not distinguish between ``blurring'' and ``churning'', terms commonly
used to refer to changes in the guiding centre radii of stars and their
epicyclic motions, respectively.
Both are induced by a variety of physical interactions such as molecular cloud
scattering~\citep{Mihalas1981, Jenkins1990, Jenkins1992}, orbital resonances
with spiral arms or bars~\citep{Sellwood2002, Minchev2011}, and satellite
perturbations~\citep*{Bird2012}.
All of these effects are included in~\hsim~and should therefore be inherited by
the stellar populations in our GCE model.
% Because our model assumes smooth migration with a~$\sqrt{\text{age}}$
% dependence (see discussion above), it captures the effects of ``churning'' but
% neglects the effects of ``blurring.'' 
% These terms are commonly used to refer to changes in the guiding centre radii
% of stars and their epicyclic motions, both of which are induced by a variety
% of physical interactions such as molecular cloud scattering
% \citep{Mihalas1981, Jenkins1990, Jenkins1992}, orbital resonances with spiral
% arms or bars~\citep{Sellwood2002, Minchev2011}, and satellite perturbations
% \citep*{Bird2012}.
% All of these effects are included in~\hsim, and therefore realistic churning
% should simply be inherited by the stellar populations in our model.
% Nonetheless, it would be straight-forward to run a version of our model with
% superimposed sinusoidal fluctuations on each stars' change in radius over
% time to mimic the effects of blurring, but we do not pursue this question
% here.
\par
We assume the SFH of the ``inside-out'' model from~\citet{Johnson2021}.
The time-dependence at a given~\rgal~is described by
\begin{equation}
\dot{\Sigma}_\star \propto (1 - e^{-t / \tau_\text{rise}})
e^{-t/\tau_\text{sfh}},
\end{equation}
where~$\tau_\text{rise}$ approximately controls the amount of time the SFR is
rising at early times; we set this parameter equal to 2 Gyr at all radii as in
\citet{Johnson2021}.
Our e-folding timescales of~$\tau_\text{sfh}$ are taken from a fit of this
functional form to the~$\Sigma_\star$-age relation in bins of~$R / R_\text{e}$
for~$10^{10.5} - 10^{11}~M_\odot$ Sa/Sb Hubble type spiral galaxies reported
by~\citet{Sanchez2020}.
The resulting values of~$\tau_\text{sfh}$ are long:~$\sim$15 Gyr at the solar
circle (\rgal~= 8 kpc) and as high as~$\sim$40 Gyr in the outer disc (see fig.
3 of~\citealp{Johnson2021}).
This is a consequence of flat nature of the~$\Sigma_\star$-age relation
reported by~\citet{Sanchez2020}.
\par
Within each~$\delta\rgal = 100$ pc ring, the normalization of the SFH is set by
the total stellar mass of the Milky Way disc and the present-day stellar
surface density gradient, assuming that it is unaffected by migration (see
Appendix B of~\citealp{Johnson2021}).
For the former, we neglect the contribution from the bulge and adopt the total
disc stellar mass of~$5.17\times10^{10}~\msun$ from~\citet{Licquia2015}.
For the latter, we adopt a double exponential form describing the thin- and
thick-disc components.
We take the scale radii of the thin- and thick-discs to be~$R_\text{t} = 2.5$
kpc and~$R_\text{T} = 2.0$ kpc, respectively, with a surface density ratio
at~\rgal~= 0 of~$\Sigma_\text{T} / \Sigma_\text{t} = 0.27$ based on the
findings of~\citet{Bland-Hawthorn2016}.

% The~\citet{Johnson2021} models run~\vice~in star formation mode, meaning that
% the user specifies the SFH and the amount of gas and infall at each timestep
% are calculated automatically by the code.
Since the~\citet{Johnson2021} models run~\vice~with the SFH specified a priori,
determining the gas supply requires an assumption regarding the SFE.
Based on the findings of~\citet{Kennicutt1998}, GCE models often adopt a
single power-law relating the surface density of gas~$\Sigma_\text{gas}$ to
the surface density of star formation~$\dot{\Sigma}_\star$.
Recent studies, however, have revealed that the spatially resolved
$\Sigma_\text{gas} - \dot{\Sigma}_\star$ relation is more nuanced than the
integrated relation~\citep{delosReyes2019, Ellison2021, Kennicutt2021}.
Some of the uncertainty regarding its details can be traced back to the
ongoing debate about the CO-to-H$_2$ conversion factor
(\citealp{Kennicutt2012};~\citealp*{Liu2015}).
Based on a compilation of the~\citet{Bigiel2010} and~\citet{Leroy2013} data
shown in comparison to the theoretically motivated parametrizations of
\citet[][see their fig. 2]{Krumholz2018},~\citet{Johnson2021} take a
three-component power-law~$\dot{\Sigma}_\star \propto \Sigma_\text{gas}^N$ with
index~$N$ given by:
\begin{equation}
N =
\begin{cases}
1.0 & (\Sigma_\text{gas} \geq 2\times10^7~\msun~\persqkpc) \\
3.6 & (5\times10^6~\msun~\persqkpc \leq \Sigma_\text{gas} \leq
2\times10^7~\msun~\persqkpc) \\
1.7 & (\Sigma_\text{gas} \leq 5\times10^6~\msun~\persqkpc).
\end{cases}
\label{eq:sflaw_indeces}
\end{equation}
The normalization is fixed by setting the SFE timescale
$\tau_\star \equiv \Sigma_\text{gas} / \dot{\Sigma}_\star$ at surface densities
where~$N = 1$ to the value derived observationally for molecular gas, denoted
by~$\tau_\text{mol}$.
This value at the present day is taken to be~$\tau_{\text{mol},0} = 2$ Gyr
\citep{Leroy2008, Leroy2013}, with a~$t^{1/2}$ dependence on cosmic time based
on the findings of~\citet{Tacconi2018} studying the properties of molecular gas
as a function of redshift.
\par
With the yields adopted in the~\citet{Johnson2021} models (see discussion at
the beginning of~\S~\ref{sec:yields}), considerable outflows are required in
order to predict empirically plausible abundances.
\citet{Weinberg2017} demonstrate analytically that the equilibrium abundance of
some element in the gas phase is approximately determined by its yield and the
mass loading factor~$\eta \equiv \dot{\Sigma}_\text{out} / \dot{\Sigma}_\star$
with a small correction for the SFH.
\citet{Johnson2021} select a scaling of~$\eta$ with~\rgal~such that the
equilibrium abundance as a function of radius corresponds to a reasonable
metallicity gradient within the Galaxy (see their fig. 3 and discussion in
their~\S~3.1).
Nonetheless, yields and the strength of outflows are mutually degenerate
parameters since they act as source and sink terms in computing enrichment
rates.
The absolute scale of nucleosynthetic yields is a topic of debate (see
discussion in, e.g.,~\citealp{Griffith2021}), and some authors even neglect
outflows entirely, arguing that they do not significantly alter the chemical
evolution of the Galaxy disc~\citep[e.g.][]{Spitoni2019, Spitoni2021}.
We investigate the impact of simultaneous variations in our yields and the
efficiency of outflows in~\S~\ref{sec:results:yields} below
{\color{red} and find similar~\ohno~relations, suggesting that we would arrive
at similar conclusions if we were to neglect outflows and adjust our GCE models
accordingly.}

\end{document}

