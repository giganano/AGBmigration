
\section{Conclusions}
\label{sec:conclusions}

Building on the multi-zone GCE model of \cite{Johnson2021}, which reproduces
many observed features of the $\feh$-$\afe$-age distribution of Milky Way
disk stars, we have inferred empirical constraints on the stellar nucleosynthesis
of N by comparing model predictions to observed gas phase trends in external
galaxies and stellar trends in the Milky Way disk.  In our models, the gas 
phase abundance at a given Galactocentric radius first evolves to higher
[O/H] at roughly constant [N/O] because of primary (metallicity-independent)
N production, then evolves upward in [N/O] with slowly increasing [O/H] because
of secondary N production that increases with metallicity (Fig.~\ref{Fig5}).
The [N/O]-[O/H] relation reaches an approximate equilibrium after $t=5-8\Gyr$,
consistent with previous arguments that this relation is largely 
redshift-independent \citep{Vincenzo2018,Hayden-Pawson2021}.
This [N/O]-[O/H] relation represents a superposition of evolutionary track
endpoints rather than an evolutionary track itself, similar to some explanations
of the low-$\alpha$ disc sequence in the Milky Way
(e.g., \citealt{Schoenrich2009,Nidever2014,Buck2020,Sharma2021,Johnson2021}).

As our principal empirical benchmark, we take Dopita et al.'s 
(\citeyear{Dopita2016}) characterization of observed gas-phase abundances
in external galaxies (see Fig.~\ref{Fig1}).  Using Johnson et al.'s
(\citeyear{Johnson2021}) CCSN oxygen yield $\yocc=0.015$, we obtain
agreement with the \cite{Dopita2016} [N/O]-[O/H] relation if we assume a 
metallicity independent massive star yield $\yncc=3.6 \times 10^{-4}$ and an
AGB fractional N yield that is linear in stellar mass and metallicity,
equation~(\ref{eqn:linear}) with $\xi = 9 \times 10^{-4}$.
This value of $\yncc$ is consistent with the rotating massive star models
of \cite{Limongi2018}, and we concur with previous arguments
(REFS including Grisoni) that rotational models are required to explain
the $\no \approx -0.7$ plateau observed at low metallicities
(Fig.~\ref{Fig2}).  The AGB yield is similar in form but $3\times$ higher
in amplitude than the models of \c11+c15.

With $\yocc = 0.015$ and $\yncc=3.6 \times 10^{-4}$, the AGB N yields of
\c11+15 or \v13 must be amplified by factors of three and two, respectively,
to achieve agreement with the \cite{Dopita2016} [N/O]-[O/H] relation
(Fig.~\ref{Fig6}).  However, as predicted abundance ratios depend primarily
on yield ratios, we can also obtain agreement by using the \c11+15 or \v13
yields and lowering $\yocc$ and $\yncc$ by the corresponding factor.
Such a change could be physically justified if black hole formation is
more extensive, or the IMF steeper, than implicitly assumed by the
value $\yocc=0.015$ (see \S\ref{sec:4.2.1} and \citealt{Griffith2021}).
Other successful predictions of the \cite{Johnson2021} models, including
the Galactic [O/H] gradient that is one of its basic constraints,
would be largely unchanged if $\yfecc$, $\yfeIa$, and outflow mass loading
efficiencies $\eta$ were all reduced by the same factor.  Alternatively,
one could retain a higher $\yocc$ and $\yncc$ but assume that galactic
winds preferentially eject CCSN products relative to AGB products, as 
suggested by \cite{Vincenzo2016}.  The degeneracy between the overall scaling
of yields and the magnitude of outflows is one of the key sources of 
uncertainty in GCE models.

In contrast to \c11+15 and \v13, the AGB models of \k10 and \kl16+k18 predict
IMF-averaged yields that are decreasing or approximately flat with 
increasing $Z$ (Fig.~\ref{Fig4}).  In our GCE models, these yields lead to
clear disagreement with the \cite{Dopita2016} trend, even when we allow
reasonable variations in the metallicity dependence of $\yncc$ (Fig.~\ref{Fig6}).
There are many physical effects in AGB stellar models, so it is difficult
to pinpoint a single cause for this discrepancy.  In general, the most
efficient N production occurs when both TDU and HBB occur simultaneously
because each replenishment of C and O isotopes from the stellar core by TDU
adds new seed nuclei for HBB to process into $^{14}$N via the CNO cycle
\citep{Ventura2013}.  The distinctive metallicity dependence of the 
\k10 and \kl16+k18 yields traces back to the simultaneous occurrence of
TDU and HBB over a substantial mass range in their low metallicity models
(Fig.~\ref{Fig3}).

All of the AGB models predict that IMF-averaged N production is dominated by
stars with $M>2 M_\odot$ (Fig.~\ref{Fig4}).  As a result, the delay time
required to produce 50\% of the AGB N is 250 Myr or less, shorter than
the $\sim 1$ Gyr characteristic delay of Fe from SNIa.  The form of the
[N/O]-[O/H] relation is driven by the metallicity dependence of N yields, 
not by the time delay of AGB production (Fig.~\ref{Fig7}).

To investigate the sources of scatter in the [N/O]-[O/H] relation, we construct
variants of our fiducial model that have $\sim 40\%$ sinusoidal oscillations
of the SFR with a 2 Gyr period, induced by oscillations in either the SFE
or the gas infall rate.  The combined effects of dilution by pristine infall
and metallicity dependent N production lead to oscillations in the 
[N/O]-[O/H] relation, with scatter at fixed [O/H] that is comparable to
that measured in MaNGA galaxies by \cite{Schaefer2020} (Fig.~\ref{Fig10}).
We concur with their conclusion that variations in SFE can plausibly explain
most of the observed scatter.  \cite{Johnson2021} find that stellar migration
induces stochastic variations in $\afe$ enrichment because a stellar population
can migrate from its birth radius before most of its SNIa Fe
production takes place.  The same effect occurs for AGB N enrichment, but
it is smaller because the shorter ($<250\Myr$) timescale provides less time
for migration.  We find that migration leads to $\sim 0.05$-dex scatter in
[N/O] at fixed [O/H], which is smaller than the scatter measured by
\cite{Schaefer2020} but not negligible.

Our findings illustrate the value and methodology of empirically constraining
stellar yields by combining general theoretical expectations with GCE modeling
and observational constraints.  For the case of nitrogen, we have used the
expectation that massive stars and AGB stars both contribute, with the AGB
contribution moderately delayed in time.  The metallicity dependence of the
combined IMF-averaged yield is tightly constrained, and it is plausibly 
partitioned into a massive star yield that is independent of metallicity and
an AGB yield that is linear in metallicity and progenitor mass.
The normalization of the yield is well constrained {\it relative} to the
IMF-averaged oxygen yield.  The delay time distribution predicted by our
fiducial model, in concert with the \cite{Johnson2021} GCE prescriptions,
leads to good agreement with the [N/O]-age and [N/O]-[O/Fe] trends for
Milky Way disk stars.  As this approach is extended to increasing numbers
of elements, the web of yield constraints and consistency tests will become
steadily more powerful, providing valuable insights on stellar astrophysics,
supernova physics, and the history of our Galaxy.
