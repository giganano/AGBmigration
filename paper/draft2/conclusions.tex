
\documentclass[ms.tex]{subfiles}
\begin{document}

\section{Conclusions}
\label{sec:conclusions}

Building on the multi-zone GCE model of~\citet{Johnson2021}, which reproduces
many observed features of the [$\alpha$/Fe]-[Fe/H]-age distribution of Milky
Way disc stars, we have inferred empirical constraints on the stellar
nucleosynthesis of N by comparing model predictions to observed gas-phase trends
in extenal galaxies and stellar trends in the Milky Way disc.
In our models, the gas-phase abundance at a given Galactocentric radius first
evolves to higher~\oh~at roughly constant~\no~because of primary
(metallicity-independent) N production, then evolves upward in~\no~with slowly
increasing~\oh~because of secondary N production that increases with
metallicity (Fig.~\ref{fig:no_oh_timeevol}).
The~\ohno~relation reaches an approximate equilibrium after~$t = 5 - 8$ Gyr,
consistent with previous arguments that this relation is largely
redshift-independent~\citep{Vincenzo2018, HaydenPawson2021}.
This~\ohno~relation represents a superposition of evolutionary track endpoints
rather than an evolutionary track itself, similar to some explanations of the
low-$\alpha$ disc sequence in the Milky Way~\citep[e.g.][]{Schoenrich2009,
Nidever2014, Buck2020, Sharma2021, Johnson2021}.
\par
As our principal observational benchmark, we take Dopita et
al.'s (\citeyear{Dopita2016}) characterization of observed gas-phase abundances
in external galaxies (see Fig.~\ref{fig:no_oh_observed}).
Using Johnson et al.'s (\citeyear{Johnson2021}) CCSN oxygen yield
of~$\ycc{O} = 0.015$, we obtain agreement with
the~\citet{Dopita2016}~\ohno~relation if we assume a metallicity-independent
massive star yield of~$\ycc{N} = 3.6\times10^{-4}$ and an AGB fractional N
yield that is linear in stellar mass and metallicity
(equation~\ref{eq:linear_yield}) with~$\xi = 9\times10^{-4}$.
This value of~$\ycc{N}$ is consistent with the rotating massive star models of
\citet{Limongi2018}, and we concur with previous arguments that rotating massive
stars are required to explain the~$\no \approx -0.7$ plateau observed at low
metallicities (see Fig.~\ref{fig:no_oh_observed};~\citealp{Chiappini2003,
Chiappini2005, Chiappini2006, Kobayashi2011, Prantzos2018, Grisoni2021}).
The AGB yield is similar in form but 3 times higher in amplitude than the
models of~\cristallo.
\par
With~$\ycc{O} = 0.015$ and~$\ycc{N} = 3.6\times10^{-4}$, the AGB N yields
of~\cristallo~and~\ventura~must be amplified by factors of three and two,
respectively, to achieve agreement with the~\citet{Dopita2016}~\ohno~relation
(Fig.~\ref{fig:no_oh_predictions}).
However, as predicted abundance ratios depend primarily on yield ratios, we can
also obtain agreement by using the~\cristallo~or~\ventura~yields and
lowering~\ycc{O}~and~\ycc{N}~by the corresponding factor.
Such a change could be physically justified if black hole formation is more
extensive, or the IMF steeper, than implicitly assumed by the value
of~$\ycc{O} = 0.015$ (see~\S~\ref{sec:results:yields:variations}
and~\citealp{Griffith2021a}).
Other successful predictions of the~\citet{Johnson2021} models, including the
Galactic~\oh~gradient that is one of its basic constraints, would be largely
unchanged if~\ycc{Fe},~\yia{Fe}, and outflow mass loading efficiencies~$\eta$
were all reduced by the same factor.
Alternatively, one could retain a higher~\ycc{O}~and~\ycc{N}~but assume that
Galactic winds preferentially eject CCSN products relative to AGB products, as
suggested by~\citet{Vincenzo2016a}.
The degeneracy between the overall scaling of yields and the magnitude of
outflows is one of the key sources of uncertainty in GCE models.
\par
In contrast to~\cristallo~and~\ventura, the AGB models
of~\karakasten~and~\karakas~predict IMF-averaged yields that are decreasing or
approximately flat with increasing~$Z$ (Fig.~\ref{fig:ssp}).
In our GCE models, these yields lead to clear disagreement with the
\citet{Dopita2016} trend, even when we allow reasonable variations in the
metallicity dependence of~\ycc{N} (Fig.~\ref{fig:no_oh_predictions}).
There are many uncertain physical effects in AGB stellar models, so it is
difficult to pinpoint a single cause for this discrepancy.
In general, the most efficient N production occurs when both TDU and HBB occur
simultaneously because each replenishment of C and O isotopes from the stellar
core by TDU adds new seed nuclei for HBB to process in~\Nfourteen~via the CNO
cycle~\citep{Ventura2013}.
The distinctive metallicity dependence of the~\karakasten~and~\karakas~yields
traces back to the simultaneous occurrence of TDU and HBB over a substantial
mass range at all metallicities (Fig.~\ref{fig:agb_yield_models}).
\par
All of the AGB models predict that IMF-averaged N production is dominated by
stars with M~$> 2~\msun$ (Fig.~\ref{fig:ssp}).
As a result, the delay-time required to produce 50\% of the AGB N is 250 Myr or
less, shorter than the~$\sim$1 Gyr characteristic delay of Fe fron SN Ia.
The form of the~\ohno~relation is driven by the metallicity dependence of N
yields, not by the time delay of AGB production (Fig.~\ref{fig:t_z_dep_comp}),
and it can be calculated accurately from simple equilibrium arguments under
most circumstances (\S~\ref{sec:results:ohno_equilibrium},
equation~\ref{eq:noeq}).
\par
\citet{Vincenzo2021} inferred the median~\ohno~trend of Milky Way disc stars
from APOGEE abundances corrected for mixing on the red giant branch using the
asteroseismic mass measurements from~\citet{Miglio2021}.
They found good agreement with the~\citet{Dopita2016} trend, our observational
benchmark, so our model is also consistent with their derived APOGEE trends.
% \citet{Vincenzo2021} find that the median~\ohno~trend of Milky Way disc stars,
% inferred from APOGEE abundances corrected for mixing on the red giant branch,
% follows the trend of~\citet{Dopita2016}, so it is also reproduced by our model.
Our model also reproduces, at least approximately, two important findings of
\citet{Vincenzo2021}:~\no~exhibits little correlation with stellar age at
fixed~\feh~for ages~$\lesssim 9$ Gyr, and~\no~declines linearly with
increasing~\ofe~at fixed~\oh~(Fig.~\ref{fig:vincenzo_comp}).
The match to these observations~\textit{does} depend on the AGB DTD, and it
breaks down if we either make the AGB enrichment instantaneous or make it occur
as slowly as SN Ia Fe production.
\par
To investigate the sources of scatter in the~\ohno~relation, we construct
variants of our fiducial model that have~$\sim$40\% sinusoidal oscillations in
the SFR with a 2 Gyr period, induced by oscillations in either the SFE or the
gas infall rate.
The combined effects of dilution by pristine infall and metallicity-dependent N
production lead to oscillations in the~\ohno~relation comparable in magnitude
to the scatter measured in MaNGA galaxies by~\citet{Schaefer2020}
(Fig.~\ref{fig:schaefer_comp}).
We concur with their conclusion that variations in the SFE can plausibly
explain most of the observed scatter.
\citet{Johnson2021} find that stellar migration induces stochastic variations
in [$\alpha$/Fe] enrichment because a stellar population can migrate from its
birth radius before most of its SN Ia Fe production takes place.
The same effect occurs for AGB N enrichment but to a lesser extent because the
shorter production timescale ($\sim$250 Myr) leaves less time for migration.
We find that migration leads to~$\sim$0.05-dex scatter in~\no~at fixed~\oh,
which is smaller than the scatter measured by~\citet{Schaefer2020} but not
negligible.
\par
Our findings illustrate the value and methodology of empirically constraining
stellar yields by combining general theoretical expectations with GCE modelling
and observational constraints.
For the case of N, we have used the expectation that massive stars and AGB stars
both contribute, with the AGB contribution moderately delayed in time.
The metallicity dependence of the combined IMF-averaged yield is tightly
constrained, and it is plausibly partitioned into a massive star yield that is
independent of metallicity and an AGB yield that is linear in metallicity~$Z$
and progenitor mass~$M$.
The normalization of the yield is well-constrained~\textit{relative} to the
IMF-averaged O yield.
The DTD predicted by our fiducial model, in concert with the~\citet{Johnson2021}
GCE prescriptions, leads to good agreement with the~\no-age and~\no-\ofe~trends
for Milky Way disc stars (Fig.~\ref{fig:vincenzo_comp}).
As this approach is extended to increasing numbers of elements, the web of
yield constraints and consistency tests will become steadily more powerful,
providing valuable insights on stellar astrophysics, SN physics, and the
history of our Galaxy.

\end{document}

