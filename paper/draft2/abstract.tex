
\documentclass[ms.tex]{subfiles}
\begin{document}

\begin{abstract}
We use a multi-ring galactic chemical evolution model to probe the
nucleosynthesis of N in the Milky Way.
This approach incorporates the effects of stellar migration by allowing stars
to move between rings in a manner based on a hydrodynamical simulation.
In order to reproduce the gas-phase~\ohno~relation, the metallicity dependence
of N yields must be approximately linear.
The combination of a massive star yield which does not depend on progenitor
metallicity with an AGB star yield which does is sufficiently accurate for
chemical evolution models.
In our model, the~\ohno~relation arises not as an evolutionary sequence but as
a superposition of endpoints.
This occurs because~\oh~reaches equilibrium significantly faster than~\nh, a
consequence of the difference in enrichment timescales between the two elements.
Our model successfully reproduces many of the observed correlations between
stellar N, O, and Fe abundances when their~\nh~measurements are corrected for
internal mixing processes known to affect the photospheric compositions of
red giant stars.
The timescale for N production is significantly shorter than the timescale for
stellar migration, and the resulting impact on enrichment rates is minimal.
We demonstrate that it is a minimal source of scatter in the
gas-phase~\ohno~relation as a consequence.
Variations in the star formation efficiency and episodic accretion both
produce changes in~\no~at fixed~\oh~that are a factor of~$\sim$2 larger than
that induced by stellar migration.
Our models run using the publicly available~\texttt{Versatile Integrator for
Chemical Evolution} (\vice;~\url{https://pypi.org/project/vice}).
\end{abstract}

\end{document}

