
\documentclass[ms.tex]{subfiles}
\begin{document}

\begin{abstract}
We derive empirical constraints on the nucleosynthetic yields of nitrogen by
incorporating N enrichment into our previously developed and empirically tuned
mutli-zone galactic chemical evolution model.
We adopt a metallicity-independent (``primary'') N yield from massive stars and
a metallicity-dependent (``secondary'') N yield from AGB stars.
In our model, galactic radial zones do not evolve along the observed
[N/O]-[O/H] relation, but first increase in [O/H] at roughly constant [N/O],
then move upward in [N/O] via secondary N production.
By~$t\approx5$ Gyr, the model approaches an equilibrium [N/O]-[O/H] relation,
which traces the radial oxygen gradient.
We find good agreement with the [N/O]-[O/H] trend observed in extra-galactic
systems if we adopt an IMF-averaged massive star yield
$y_\text{N}^\text{CC}=3.6\times10^{-4}$, consistent with predictions for
rapidly rotating progenitors, and a fractional AGB yield that is linear in mass
and metallicity
$y_\text{N}^\text{AGB}=(9\times10^{-4})(M_*/M_\odot)(Z_*/Z_\odot)$.
This model reproduces the [N/O]-[O/H] relation found for Milky Way stars
in the APOGEE survey as well as (though imperfectly) the trends of
stellar [N/O] with age and [O/Fe].
The metallicity-dependent yield plays the dominant role in shaping the gas-phase
[N/O]-[O/H] relation, but the AGB time-delay is required to match the APOGEE
stellar age and [O/Fe] trends.
If we add~$\sim$40\% oscillations to the star formation rate, the model
reproduces the scatter in gas-phase [N/O] vs. [O/H] observed in external
galaxies by MaNGA.
We also construct models using published AGB yields and examine their empirical
successes and shortcomings.
For all AGB yields we consider, simple stellar populations release half their
N after only~$\sim$250 Myr.
\end{abstract}

\end{document}

