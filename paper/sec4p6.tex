
\subsection{The [N/O]-[O/H] relation as an equilibrium phenomenon}

As shown by \cite{Weinberg2017}, under generic conditions the abundances of
a one-zone model with continuing gas infall evolve to an equilibrium, in 
which new metal production is balanced by dilution and by the loss of ISM
metals into stars and outflows.  Fig.~\ref{Fig5} shows that the [N/O] ratio
at a given radius in our multi-zone GCE model approaches equilibrium
after $t \approx 5\Gyr$.  Fig.~\ref{Fig7} further shows that the [N/O]-[O/H]
relation that emerges in our model is driven by the metallicity dependence
of the N yield with the time delay of AGB enrichment having minimal impact.
It is therefore useful to apply insights from analytic models to the origin
of this relation, extending our discussion from \S\ref{Sec4.2.1}.

For an exponential star formation history 
$\mdotstar \propto e^{-t/\tau_{\rm SFH}}$ and instantaneous enrichment and
recycling of an element with IMF-averaged yield $y$, the equilibrium ISM
mass fraction is
\begin{equation}
Z_{\rm eq} = y/(1+\eta-r-\taustar/\tau_{\rm SFH}~,
\label{eqn:zeq}
\end{equation}
where, as before $\eta = \dot{M}_{\rm out}/\mdotstar$, $r \approx 0.4$ is
the recycling fraction, and $\taustar = M_{\rm gas}/\mdotstar$ is the 
SFE timescale \citep{Weinberg2017}.  When an element's characteristic
enrichment time delay is $\ll \tau_{\rm SFH}$, the correction to 
equation~(\ref{eqn:zeq}) is very small.  Therefore, even with a declining
SFR and AGB time delay, the N/O abundance ratio in equilibrium should equal
the yield ratio.  While a full time-dependent analytic solution is only
possible for a metallicity-independent yield, to compute the equilibrium ratio 
we can just take the N yield at the metallicity implied by the oxygen 
abundance.  This argument leads to the expectation
\begin{equation}
10^{[N/O]} = {(Z_N/Z_O}_{\rm eq} \over (Z_{N,\odot}/Z_{O,\odot})} 
           = {y_N(Z_O)/y_O \over Z_{N,\odot}/Z_{O,\odot}}~,
\label{eqn:NOeq}
\end{equation}
with $10^{[O/H]} = Z_O/Z_{O,\odot}$.

Equation~\ref{eqn:NOeq} can be used to predict the [N/O]-[O/H] relation
for a given set of yields.  {\bf TBD:} We have spot-checked this formula
against a number of the cases shown in Fig.~\ref{Fig6} and find agreement
to 0.05 dex {\bf TBD} or better.  One can also use equation~(\ref{eqn:NOeq})
to reverse engineer the IMF-averaged N yield required to match an
empirical [N/O]-[O/H] relation for a specified $y_O$.  Our fiducial
AGB yield (equation XX with $\xi = 9\times 10^{-4}$ gives an IMF-averaged
yield of $\ynagb = 9.4\times 10^{-4}$ {\bf Check.  Read off plot.}
at $Z = Z_\odot$, with a linear dependence on $Z$.
Applying this yield (and $\yncc = 3.6 \times 10^{-4}$, $\yocc = 0.015$,
$Z_{N,\odot}=6.91 \times 10^{-4}$, $Z_{O,\odot} = 5.72 \times 10^{-3}$)
to equation~(\ref{eqn:NOeq}) gives
\begin{equation}
10^{[N/O]} = 0.199 + 0.519 \times 10^{[O/H]}~.
\end{equation}

ONE MORE PARAGRAPH ON THE LIMITS OF EQUILIBRIUM.

