
\documentclass[ms.tex]{subfiles}
\begin{document}

\section{The Multi-Zone Chemical Evolution Model}
\label{sec:multizone}

We use the fiducial chemical evolution model for the Milky Way published in
\citet{Johnson2021}, which runs using the~\vice~chemical evolution code
(\citealp{Johnson2020, Griffith2021a}; see Appendix~\ref{sec:vice}).
\citet{Johnson2021} focus their discussion on the predicted O and Fe
abundances, but because~\vice~recognizes most elements on the periodic table,
computing N abundances with this model is easy.
Multi-zone models allow simultaneous calculations of enrichment rates and
abundances for multiple Galactic regions, making them a more appealing option
than classical one-zone models for a system like the Milky Way.
Furthermore, they can take into account stellar migration in a framework that
is much less computationally expensive than hydrodynamical simulations, making
them the ideal experiments for our purposes.
We provide a brief summary of the model here, but a full breakdown can be found
in~\S~2 of~\citet{Johnson2021}.
\par
As in previous models for the Milky Way~\citep[e.g.][]{Matteucci1989,
Schoenrich2009, Minchev2013, Minchev2014, Minchev2017, Sharma2021}, this model
parameterizes the Galaxy disc as a series of concentric rings of uniform width
$\delta\rgal = 100$ pc.
Each ring is assigned its own star formation history (SFH), and with
assumptions about the~$\Sigma_\text{gas}-\dot{\Sigma}_\star$ and outflows (see
discussion below),~\vice~computes the implied amounts of gas and infall at each
timestep automatically.
Under the caveat that stellar populations can move between rings and place
their nucleosynthetic products in a ring other than the one they were born in,
each ring is otherwise described by a conventional one-zone GCE model.
Tracking enrichment as stars migrate was a novel addition by the
\citet{Johnson2021} model which they demonstrate to have a significant impact
on the enrichment rates from delayed sources such as SNe Ia.
\par
To drive stellar migration, the model makes use of star particles from a
hydrodynamical simulation, for which~\citet{Johnson2021} select the~\hsim~galaxy
from the~\citet{Christensen2012} suite evolved with the N-body+SPH code
\texttt{GASOLINE}~\citep[][SPH: Smoothed Particle Hydrodynamics]{Wadsley2004};
we retain this decision here.
The star particles with reliable birth radii from~\hsim~span a range of 13.2
Gyr in age, so our model places the onset of star formation~$\sim$500 Myr
after the big bang and integrates up to the present day.
Previous studies have shown that~\hsim, among other disc galaxies evolved with
similar physics, has a realistic rotation curve~\citep{Governato2012,
Christensen2014a, Christensen2014b}, stellar mass~\citep{Munshi2013},
metallicity~\citep{Christensen2016}, dwarf satellite population
\citep{Zolotov2012, Brooks2014}, HI properties~\citep{Brooks2017}, and stellar
age-velocity relation~\citep{Bird2021}.
Despite this, there are some interesting differences between~\hsim~and the
Milky Way.
First and foremost,~\hsim~had only a weak and transient bar, lacking one at the
present day, while the Milky Way is known to have a strong, long-lived central
bar~\citep[e.g.][]{Bovy2019}.
This could indicate that the dynamical history of~\hsim~and its star particles
differs significantly from that of the Milky Way.
Furthermore, the last major merger in~\hsim~was at a redshift of~$z \approx 3$,
making it an interesting case study for its quiescent merger history
\citep[e.g.][]{Zolotov2012}, while the Sagitarrius dwarf galaxy, a relatively
massive satellite, is believed to have made pericentric passages around the
Milky Way at~$1 - 2$ Gyr intervals~\citep{Law2010}.
With this in mind, a dynamical history such as that of~\hsim~in this GCE model
can be thought of as capturing purely secular galaxy evolution.
Although these differences between~\hsim~and the Milky Way are well understood,
their impact on chemical evolution is not.
We are unaware of any studies which investigate the impact of different
assumptions regarding the Galaxy's dynamical history and the subsequent effects
on predicted abundances; this would be an interesting direction for future work.
\par
Radial migration of stars proceeds from the~\hsim~star particles in a simple
manner; for a stellar population in our model born at a radius~\rgal~and a
time~$T$,~\vice~searches for star particles born at~$\rgal \pm 250$ pc and
$T \pm 250$ Myr.
From the star particles that make this cut, it then randomly selects one to act
as that stellar population's~\textit{analogue}.
The stellar population then assumes the present day midplane distance~$z$ and
the change in orbital radius~$\Delta\rgal$ of its analogue.
In the~\citet{Johnson2021} fiducial model, stellar populations move to their
implied final radii with a~$\sqrt{\text{age}}$ dependence, similar to the
assumption made by~\citet{Frankel2018, Frankel2019}.
Although they investigate other assumptions for this time-dependence, in the
present paper we make use of only this model (hereafter referred to as the
``diffusion'' model) and an idealized one in which stars remain at their birth
radius and then instantaneously migrate at the present day (hereafter referred
to as the ``post-processing'' model).
If~\vice~does not find any star particles from~\hsim~in its initial
$\rgal \pm 250$ pc and~$T \pm 250$ Myr search, it widens it to
$\rgal \pm 500$ pc and~$T \pm 500$ Myr; if still no candidate analogues are
found,~\vice~maintains the~$T \pm 500$ Myr requirement, but assigns the star
particle with the smallest difference in birth radius as the analogue.
This procedure can be thought of as ``injecting'' the dynamics of
the~\hsim~galaxy into our multi-zone chemical evolution model, and can in
principle be repeated for any hydrodynamical simulation of a disc galaxy.
As in~\citet{Johnson2021}, we neglect radial gas flows
\citep[e.g.][]{Lacey1985, Bilitewski2012, Vincenzo2020}, instead focusing on
the impact of stellar migration.
\par
Although this model does impose some small but nonzero level of star formation
at early times in the outer disc, the sample of star particles from~\hsim~is
sufficiently large such that stellar populations that form there are typically
assigned analogues which formed within~$\sim$2 kpc of their birth radius.
While ignoring effects such as the radial growth of the Galaxy (e.g.
\citealp*{Bird2012};~\citealp{Bird2013}), this at least ensures that these old,
outer disc populations are assigned star particles which given them an outer
disc rather than an inner disc dynamical history.
\par
Rather than using a hydrodynamical simulation, some previous studies have
implemented stellar migration using dynamical arguments
\citep[e.g.][]{Schoenrich2009, Sharma2021}.
An advantage of our approach over this is that these dynamical arguments
introduce free parameters into the modelw hich then require fitting to data.
It is also unclear the extent to which fitting to observed data biases the
model into agreement with parts of the sample not involved in the fit.
A disadvantage is that we are restricted to one realization of our dynamical
history; slight variations are not possible.
We do not distinguish between ``blurring'' and ``churning'': terms often used
to refer to the epicyclic motions of stars and changes in their guiding centre
radii, respectively.
These effects are induced by a variety of physical interactions such as
molecular cloud scattering~\citep{Mihalas1981, Jenkins1990, Jenkins1992},
orbital resonances with spiral arms or bars~\citep{Sellwood2002, Minchev2011},
and satellite perturbations~\citep{Bird2012}.
All of these effects are present in~\hsim~and should therefore be inherited to
some extent by the stellar populations in our GCE model.
\par
Our fiducial model here has the same SFH as that of~\citet{Johnson2021}, where
the time-dependence at a given~\rgal~is given by
\begin{equation}
\dot{\Sigma}_\star \propto (1 - e^{-t / \tau_\text{rise}})
e^{-t/\tau_\text{sfh}},
\end{equation}
where~$\tau_\text{rise}$ approximately controls the amount of time the SFR is
rising at early times; we set this parameter equal to 2 Gyr at all radii as in
\citet{Johnson2021}.
Our e-folding timescales of~$\tau_\text{sfh}$ are taken from a fit to this
functional form to the~$\Sigma_\star$-age relation in bins of~$R / R_\text{e}$
for~$10^{10.5} - 10^{11}~M_\odot$ Sa/Sb Hubble type spiral galaxies reported
by~\citet{Sanchez2020}.
The resulting values of~$\tau_\text{sfh}$ are long:~$\sim$15 Gyr at the solar
circle (\rgal~= 8 kpc) and as high as~$\sim$40 Gyr in the outer disc (see Fig.
3 of~\citealp{Johnson2021}).
This is a consequence of flat nature of the~$\Sigma_\star$-age relation
reported by~\citet{Sanchez2020}.
\par
Within each~$\delta\rgal = 100$ pc ring, the normalization of the SFH is set by
the total stellar mass of the Milky Way disc and the present-day surface
density gradient assuming it is unaffected by stellar migration (see Appendix
B of~\citealp{Johnson2021}).
For the former, we neglect the contribution from the bulge and adopt the total
disc stellar mass of~$5.17\times10^{10}~M_\odot$ from~\citet{Licquia2015}.
For the latter, we adopt a double exponential form describing the thin- and
thick-disc components.
We take the scale radii of the thin- and thick-discs to be~$R_\text{t} = 2.5$
kpc and~$R_\text{T} = 2.0$ kpc, respectively, with a surface density ratio
at~\rgal~= 0 of~$\Sigma_\text{T} / \Sigma_\text{t} = 0.27$ based on the
findings of~\citet{Bland-Hawthorn2016}.
\par
The~\citet{Johnson2021} models run~\vice~in star formation mode, meaning that
the user specifies the SFH and the amount of gas and infall at each timestep
are calculated automatically by the code.
Determining the gas supply requires an assumption about the star formation 
law (often referred to as ``star formation efficiency'' in the chemical
evolution literature, though this term often has other meanings in, e.g., the
star formation and feedback community).
GCE models have and often still do adopt a single power-law relating the
surface density of gas~$\Sigma_\text{gas}$ to the surface density of star
formation~$\dot{\Sigma}_\star$ based on the findings of~\citet{Kennicutt1998}.
Recent studies, however, have revealed that the star formation law on a
galaxy-by-galaxy basis is much more nuanced~\citep{delosReyes2019, Ellison2021,
Kennicutt2021}.
Some of the uncertainty regarding its details can be tracked back to the
ongoing debate about the CO-to-H$_2$ conversion factor (\citealp{Kennicutt2012};
\citealp*{Liu2015}).
Based on a compilation of the~\citet{Bigiel2010} and~\citet{Leroy2013} data
shown in comparison to the theoretically motivated star formation laws of
\citet[][see their Fig. 2]{Krumholz2018},~\citet{Johnson2021} take a
three-component power-law as their star formation law with index given by:
\begin{equation}
N =
\begin{cases}
1.0 & (\Sigma_\text{gas} \geq 2\times10^7~\msun~\persqkpc) \\
3.6 & (5\times10^6~\msun~\persqkpc \leq \Sigma_\text{gas} \leq
2\times10^7~\msun~\persqkpc) \\
1.7 & (\Sigma_\text{gas} \leq 5\times10^6~\msun~\persqkpc).
\end{cases}
\label{eq:sflaw_indeces}
\end{equation}
The normalization of the star formation law is then set by letting the SFE
timescale~$\tau_\star \equiv \Sigma_\text{gas} / \dot{\Sigma}_\star$ be given
by the value derived observationally for molecular gas at surface densities
where~$N = 1$.
The value of~$\tau_\star$ for molecular gas at the present day is taken to be
$\tau_{\text{mol},0} = 2$ Gyr based on~\citet{Leroy2008, Leroy2013} with a
$t^{1/2}$ time-dependence based on the findings of~\citet{Tacconi2018} studying
the~$\Sigma_\text{gas} - \dot{\Sigma}_\star$ relation as a function of
redshift.
\par
Because of the yields adopted in the~\citet{Johnson2021} models, considerable
outflows are required in order to predict plausible abundances.
\citet{Weinberg2017} demonstrate analytically that to first order the
equilibrium abunadnce of some element in the gas phase is determined by its
yield and the mass loading factor
$\eta = \dot{\Sigma}_\text{out} / \dot{\Sigma}_\star$ with a small correction
for the SFH.
\citet{Johnson2021} make use of this to select a scaling of~$\eta$
with~\rgal~such that the equilibrium abundance as a function fo radius
corresponds to a reasonable metallicity gradient within the Galaxy (see their
Fig. 3 and discussion in their~\S~3.1).
Nonetheless, one can lower all yields and~$\eta$ at all~\rgal~by similar
factors, and with all other model parameters held fixed, GCE models in general
make similar predictions.
The absolute scale of nucleosynthetic yields is a topic of debate (see
discussion in, e.g.,~\citealp{Griffith2021a}), and some authors argue that
outflows do not significantly alter the chemical evolution of the Galaxy disc
and neglect them entirely~\citep[e.g.][]{Spitoni2019, Spitoni2021}.
We investigate the impact of simultaneous variations in our yields and the
efficiency of outflows in our models in~\S~\ref{sec:results:yields} below.

\end{document}

