
\documentclass[ms.tex]{subfiles}
\begin{document}

\begin{abstract}
We use a multi-ring galactic chemical evolution model to probe the
nucleosynthesis of nitrogen (N) in the Milky Way.
This approach treats individual annuli in the Galaxy disc as conventional
one-zone models, allowing stars to move between rings in a manner based on a
hydrodynamical simulation to mimic the effects of stellar migration.
We find that in order to reproduce the gas-phase~\ohno~relation as observed,
N yields must scale roughly linearly with the initial metallicity of the
progenitor stellar populations.
Of the previously published models for N nucleosynthesis in asymptotic giant
branch stars which predict this scaling, their normalization is correct if and
only if a substantial fraction of high mass stars collapse directly to black
holes rather than exploding as supernovae.
Otherwise, we must artificially amplify these N yields to offset the additional
oxygen.
Our model successfully reproduces many of the observed correlations between
stellar N, O, and Fe abundances when their~\nh~measurements are corrected for
internal mixing processes known to affect the photospheric compositions of
evolved stars.
With all of our yield models, N production timescales are sufficiently short
such that stellar migration is only a minimal source of intrinsic scatter in
the observed gas-phase~\ohno~relation.
Typical variations in the star formation efficiency produce variations
in~\no~at fixed~\oh~that are a factor of~$\sim$2 larger than that induced by
migration.
Our models run using the publicly available~\texttt{Versatile Integrator for
Chemical Evolution} (\texttt{VICE};~\url{https://pypi.org/project/vice}).
\end{abstract}

\end{document}

