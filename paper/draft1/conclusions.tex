
\documentclass[ms.tex]{subfiles}
\begin{document}

\section{Conclusions}
\label{sec:conclusions}

We have made use of the GCE models from~\citet{Johnson2021} which characterize
the Milky Way disc as a series of concentric rings with a uniform width
$\delta\rgal = 100$ pc.
As in previous models with similar motivations~\citep{Matteucci1989,
Wyse1989, Prantzos1995, Schoenrich2009, Minchev2013, Minchev2014, Minchev2017,
Sharma2021}, this model treats each individual ring as a conventional one-zone
model of chemical evolution.
The novel addition to this model, however, is that it takes into account the
impact that radial migration has on enrichment rates by allowing stellar
populations to enrich a distribution of radii as they migrate.
We retained the IMF-averaged SN yields of O and Fe from~\citet{Johnson2021},
who in turn base these values off of~\citet{Weinberg2017} and
\citet{Johnson2020}.
Assuming that the N yield from SNe Ia is negligible, we have investigated
theoretical and empirical N yields from both massive stars and AGB stars.
\par
If our fiducial CCSN yield of O is accurate ($\ycc{O} = 0.015$), then the CCSN
yield of N required to reproduce the ``plateau'' of~$\no\subcc = -0.7$ at
low~\oh~is~$\ycc{N} = 3.6\times10^{-4}$.
Out of a handful of CCSN nucleosynthesis investigations
\citep[e.g.][]{Woosley1995, Nomoto2013, Sukhbold2016}, only~\citet{Limongi2018}
present yields for rotating progenitors.
With the non-rotating models falling short of this value (in some cases by
multiple orders of magnitude; see Fig.~\ref{fig:n_cc_yields}), we argue that
the effects of rotation are necessary to explain the N abundances seen at
low~\oh, consistent with recent results from~\citet{Grisoni2021}.
\par
Various AGB star nucleosynthesis models predict vastly different yields of N
as a function of progenitor mass and metallicity (see
Fig.~\ref{fig:agb_yield_models}).
Ascertaining the origin of these differences is difficult because each model
folds in different assumptions regarding important evolutionary parameters such
as mass loss, opacities of various isotopic species, convection and convective
boundaries, and nuclear reaction networks (see discussion in, e.g.,~\S~5
of~\citealp{Karakas2016}).
Nonetheless the differences between yield models can be qualitatively
understood by considering the differences between how TDU and HBB proceed in
the stellar models (see discussion in~\S~\ref{sec:yields:agb}).
In general, the most efficient N production occurs when both TDU and HBB
occur simultaneously because each replenishment of C and O isotopes from the
stellar core by TDU add new seed nuclei for HBB to process into~\Nfourteen~via
the CNO cycle~\citep{Ventura2013}.
Different assumptions regarding mass loss and opacity can also impact the
duration of a star's AGB phase, inducing secondary effects on the N yields.
\par
When weighted by the IMF~\citep{Kroupa2001}, the~\cristallo~yields predict
roughly mass-independent N yields.
Each additional model shows a substantially higher contribution from AGB stars
with masses of~$\gtrsim$3~\msun.
In all cases, the characteristic delay times for production by single stellar
populations are~$\lesssim$250 Myr.
This is true even for the~\cristallo~yields because of the steep nature of the
stellar mass-lifetime relation: most of the mass range of stars that will go
through an AGB phase within the age of the universe will do so within a few
hundred Myr of their formation.
This characteristic delay time only applies to N because AGB star yields
generally vary from element to element, and N is somewhat unique in that its
highest yields are from AGB stars of masses~$M_\text{ZAMS} \gtrsim 4~\msun$.
\par
With a smooth SFH, we find that our model predicts the gas-phase~\ohno~relation
to be relatively time-independent up to lookback times of~$\sim6 - 8$ Gyr.
This is consistent with previous arguments that the relation is largely
redshift-independent~\citep{Vincenzo2018, HaydenPawson2021}.
Similar to previous theoretical results regarding the low [$\alpha$/Fe] stars
in the Milky Way disc~\citep[e.g.][]{Schoenrich2009, Nidever2014, Buck2020,
Sharma2021, Johnson2021}, we find that the~\ohno~relation arises not out of an
evolutionary sequence but as a superposition of endpoints.
That is, the time evolution of each Galactic region through the~\ohno~plane
is not the same line as the~\ohno~relation that would be observed at the
present day in our model Galaxy (see Fig.~\ref{fig:no_oh_timeevol}).
\par
Our fiducial model requires a renormalization of the AGB star yields of N in
order to reproduce the~\ohno~relation as observed.
The~\cristallo~and~\ventura~yield models require an artificial amplification by
factors of 2 and 3, respectively.
However, these yield models are successful if we instead simultaneously lower
our SN yields from~\citet{Johnson2021} and the outflow mass loading factor
$\eta$ at all radii by similar factors (see Fig.~\ref{fig:no_oh_predictions}).
This is physically plausible if a substantial fraction of high mass stars
collapse to black holes instead of ending their lives as CCSNe.
Although there is presently no combination of a massive star nucleosynthesis
model and a physically motivated black hole landscape able to reproduce the
observed abundance patterns~\citep{Griffith2021a}, extensive black hole
formation still lowers SN yields by simply robbing the ejecta of the explosive
nucleosynthesis component.
This suggests that either N nucleosynthesis in AGB stars must be more efficient 
or that a substantial fraction of~$M \gtrsim 8~\msun$ stars must produce
failed supernova; the answer may also be a combination of the two.
\par
With both the~\karakasten~and~\karakas~AGB star yields, our model is
unable to reproduce the monotonic increase of~\no~with~\oh~as observed.
This discrepancy persists if we consider alternate parameterizations of the
CCSN yield of N~\ycc{N}~as suggested by the non-rotating models of
\citet{Woosley1995},~\citet{Nomoto2013},~\citet{Sukhbold2016}, and
\citet{Limongi2018}.
In general, we find that reproducing the~\ohno~relation requires the total N
yield from CCSNe and AGB stars to scale roughly linearly with the progenitor
metallicity~$Z$.
We have the most success with a metallicity independent~\ycc{N}~as suggested
by CCSN models with rotating progenitors to set the zero-point and an AGB
star yield of N which scales linearly with~$Z$ as in the right panel of
Fig.~\ref{fig:ssp}.
The normalization of the AGB star yields, however, depends on the SN yields and
the efficiency of outflows as discussed above and
in~\S~\ref{sec:results:yields}.
\par
To test our model against N abundances observed in stars, we make use of the
measurements presented in~\citet{Vincenzo2021}.
To estimate birth abundances of N, they use~\texttt{MESA} stellar evolution
models~\citep{Paxton2011, Paxton2013, Paxton2015, Paxton2018} to correct the
spectroscopically derived measurements for internal mixing processes known to
affect the photospheric abundances of N in evolved stars (i.e. dredge-up of
CNO cycle products;~\citealp{Gilroy1989, Korn2007, Lind2008, Souto2018,
Souto2019}).
They find that~\no~at fixed~\feh~is relatively age-independent, a result
which our model successfully reproduces (see Fig.~\ref{fig:vincenzo_comp}).
This is a notable success of our model and the~\citet{Vincenzo2021} estimates
because with uncorrected N abundances,~\no~at fixed~\feh~shows an inverse
dependence on stellar age (see their Fig. 7).
Additionally, our model predicts~\no~to increase with decreasing~\ofe, a
result which~\citet{Vincenzo2021} derive for both corrected and uncorrected
N measurements, suggesting that the chemical dichotomy between the thin and
thick discs persists even when using birth abundances of N.
Both of these results arise out of a correlation between~\nh~and~\feh~in the
ISM predicted by the model.
Although the characteristic delay time for N production from a single stellar
population is only~$\sim$250 Myr, metallicity dependent yields dictate that
more abundant species like O must be produced in substantial amounts before N
yields from AGB stars become significant.
N production from many stellar populations is consequently more delayed than
one might expect given the production timescales from a single population.
As a result, the model predicts~\nh~to correlate much more with~\feh~than~\oh~up
to lookback times of~$\sim$10 Gyr.
This gives rise to both the flat nature of the~\no-age relation at
fixed~\feh~and the inverse relationship between~\no~and~\ofe~(see
Fig.~\ref{fig:vincenzo_comp} and discussion
in~\S~\ref{sec:results:vincenzo_comp}).
\par
To investigate the sources of scatter in the~\ohno~relation, we construct two
variations of the fiducial model from~\citet{Johnson2021}.
In these alternate scenarios, we impose 25\% sinusoidal oscillations on one of
the SFE or the SFR as functions of time while incorporating the effects of
stellar migration on the enrichment rates.
{\color{red}
These oscillations are characteristic of what we see in~\hsim, the galaxy from
which our model's dynamical history is drawn (potentially swap this out for
observational references).
}
In general, we find that these 25\% oscillations induce variability in the
gas-phase~\no~ratio at fixed~\oh~that is a factor of~$\sim$2 larger than that
caused by stellar migration.
This is a consequence of the quick production timescale of N by single stellar
populations ($\sim$ 250 Myr, see Fig.~\ref{fig:ssp} and discussion
in~\S~\ref{sec:yields:imf_agb}): there simply is not much time for orbits to
dynamically evolve before most of a stellar population's N is ejected to the
ISM.
The change in~\no~at fixed~\oh~caused by our oscillatory models is comparable
to the scatter in the relation derived observationally by~\citet{Schaefer2020}
using data from the MaNGA IFU survey~\citep{Bundy2015}.
They demonstrate that this scatter is correlated with variations in the local
SFE, with lower SFE systems exhibiting higher~\no~at fixed~\oh.
Although~\citet{Schaefer2020} could not rule out migration as an additional
source of scatter, this supports their argument that local variations in the
SFE are the primary driver, additionally suggesting that migration plays a
sub-dominant but nonetheless significant role.
\par
The small impact of stellar migration on N enrichment rates is the opposite of
what~\citet{Johnson2021} find for SN Ia enrichment of Fe.
Because the SN Ia DTD has a characteristic delay time closer to~$\sim$1 Gyr
and a significant tail to longer values, a substantial fraction of Fe
production occurs on timescales similar to the migration timescale in this
model.
Consequently, the impact on enrichment rates is much stronger for Fe (as high
as a factor of~$\sim$3 at~$\rgal \gtrsim 9$ kpc; see discussion in~\S\S~3.1 and
3.4 of~\citealp{Johnson2021}).
This is a sufficiently strong effect such that it could explain the
intrinsically young sub-component of the young~$\alpha$-rich stars observed in
the solar neighbourhood~\citep{Chiappini2015, Martig2015, Martig2016,
Jofre2016, Yong2016, Izzard2018, SilvaAguirre2018, Warfield2021}.
Although our model suggests that the impact may be larger when the Galaxy was
young, this effect is only at the~$\sim$0.05 dex level for N (see
Fig.~\ref{fig:no_oh_timeevol} and discussion in~\S~\ref{sec:results:fiducial}).
This difference underscores the argument from~\citet{Johnson2021} that in order
for nucleosynthetic yields to migrate along with their progenitor stellar
populations, the characteristic delay time for the enrichment of some element
from a single stellar population must be at least comparable to the timescales
of stellar migration.
\par
The results outlined in this paper highlight the importance of empirically
calibrated yields of all elements from all nucleosynthetic sources in GCE
models.
The combination of theoretically predicted yields and flexible computational
tools such as~\vice~can provide powerful constraints for future models of
elemental production and galaxy evolution.
When future spectroscopic surveys begin collecting data, studies such as this
will be essential to illuminating the lessons they have to teach us.

\end{document}

