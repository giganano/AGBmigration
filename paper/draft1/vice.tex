
\documentclass[ms.tex]{subfiles}
\begin{document}

\section{\vice}
\vice\footnote{
	Install (PyPI):~\url{https://pypi.org/project/vice} \\
	Documentation:~\url{https://vice-astro.readthedocs.io} \\
	Source Code:~\url{https://github.com/giganano/VICE.git}
} is an open-source~\python~package designed to model chemical enrichment
processes in galaxies with a generic, flexible model.
With this paper, we mark the release of version 1.3.0 which presents a handful
of new features:
\begin{enumerate}
	\item Users may select a mass-lifetime relation for stars from a list of
	several parameterized forms taken from the literature.
	Previously, only a single power-law was implemented, but this formulation
	underestimates lifetimes for stars with masses~$\gtrsim 4 M_\odot$; now,
	the options include the equations presented in:
	\begin{itemize}
		\item \citet{Vincenzo2016b}

		\item \citet*{Hurley2000}

		\item \citet{Kodama1997}

		\item \citet{Padovani1993}

		\item \citet{Maeder1989}

		\item \citet{Larson1974}~\textbf{(default)}
	\end{itemize}
	Generally, chemical evolution models make similar predictions with each of
	these different forms of the mass-lifetime relation since their
	quantitative predictions are not considerably different from one another
	(see the section titled ``Single Stellar Populations'' under~\vice's
	science documentation for further discussion\footnote{\url
	{https://vice-astro.readthedocs.io/en/latest/science_documentation/}
	}).
	We select the~\citet{Larson1974} form as a default within~\vice~because it
	is typical compared to the others and requires the lowest amount of
	computational overhead (aside from the single power-law option).

	\item We have added two additional tables of AGB star yields sampled at
	various progenitor masses and metallicities: 
	the~\karakas~and~\ventura~models presented in this paper are new
	to~\vice~(see discussion in~\S~\ref{sec:yields:agb} for details).

	\item We have built in the SN Ia yields presented in~\citet{Gronow2021a,
	Gronow2021b}.
	These tables present yields for double detonations of sub-Chandrasekhar
	mass carbon-oxygen white dwarfs at various progenitor metallicities.
\end{enumerate}
Although~\vice~includes built in SN and AGB star yield tables, users are not
required to adopt any one of them for use in their chemical evolution models.
Instead, it allows arbitrary functions of metallicity for both CCSN and SN Ia
yields and functions of progenitor mass and metallicity for AGB star yields.
It provides similar flexibility for additional parameters typically built into
GCE models.
\vice's backend is implemented entirely in ANSI/ISO~\texttt{C}, providing it
with the powerful computing speeds of a compiled library while retaining such
scientific flexibility within the easy-to-use framework of~\python.
\par
Requiring a Unix kernel,~\vice~supports Mac and Linux operating systems;
Windows users should install and use~\vice~entirely within the Windows
Subsystem for Linux.
On machines with x86\_64 hardware, it can be installed in a terminal via
\texttt{pip install vice}.
Users running ARM64 hardware (e.g. Macintosh computers with Apple's new M1
processor) must install~\vice~by compiling from source,
instructions for which can be found in the documentation.
After installing, running~\texttt{vice -{}-docs} and
\texttt{vice -{}-tutorial} from a~\texttt{Unix} terminal will launch a web
browser to the documentation and to a jupyter notebook intended to familiarize
first time users with~\vice's API.


\end{document}
