
\documentclass[ms.tex]{subfiles} 
\begin{document} 

\section{Introduction} 
\begin{itemize} 
	\item Nitrogen (N) is an element that traces slow neutron capture 
	(s-process) nucleosynthesis. To first order it's produced only in core 
	collapse supernovae (CCSNe) and asymptotic giant branch (AGB) stars 
	\citep{Johnson2019}. 

	\item Nitrogen has considerable yields through~\textit{secondary} channels: 
	the processing of already produced metals into nitrogen. 
	\begin{itemize} 
		\item First and foremost is the CNO cycle, in which carbon (C), N, and 
		oxygen (O) catalyze the fusion of four protons into helium-4. The 
		reactions of the CNO cycle: 
		\begin{equation} 
		\Ctwelve(p,\gamma)\Nthirteen(\beta^+,\nu_e)\Cthirteen 
		(p,\gamma)\Nfourteen(p,\gamma)\Ofifteen(\beta^+,\nu_e)
		\Nfifteen(p,\alpha)\Ctwelve 
		\end{equation} 
		Due to a small cross section for proton capture, the 
		$\Nfourteen(p,\gamma)\Ofifteen$ reaction is particularly slow. 
		As a result, to first order the effect of the CNO cycle is to process 
		all of the available C and O into~\Nfourteen. 
	\end{itemize} 
\end{itemize} 

\end{document} 

