
\documentclass[12pt]{article}
\usepackage[margin = 1in]{geometry}
\usepackage{xcolor}
\pagestyle{empty}
\newcommand\doublebreak[0]{\par\null\par\noindent}

\begin{document}

\noindent
Dear Editor,
\par\null\par\noindent
We thank both reviewers for their time and careful consideration of our
submitted manuscript titled~\textit{Empirical Constraints on the
Nucleosynthesis of Nitrogen.}
We have taken their responses into consideration and updated the document
accordingly; changes from the submitted version are highlighted in red text.
Below you will find our separate responses to both reviewers' comments, with
relevant excerpts of their responses in italics.
\par\null\par\noindent
Sincerely,
\\
James W. Johnson, on behalf of the authors
\par\noindent
\begin{center}
\makebox[\linewidth]{\rule{0.5\textwidth}{0.4pt}}
\end{center}
\noindent
\begin{center}
\textbf{Reviewer 1}
\end{center}
\par\noindent
\textit{%
First, the suggested empirical yields depend on the assumptions of the adopted
GCE model, especially the IMF shape and the outflow strength (but is an outflow
really needed?).
It is unlikely that other GCE models, resting on different assumptions, will
benefit from the adoption of the proposed empirical yields...
Detailed yields from full stellar evolution studies are available nowadays from
many different groups.
}
\doublebreak
The reviewer misunderstands the motivation of this paper.
Our central result has important implications for stellar evolution models.
Nonetheless, analytic parametrizations of elemental yields in GCE models are
still useful for their flexibility and may an attractive option given the
discrepancies between predicted yields.
\doublebreak
\textit{%
Indeed, I am quite surprise that the authors need to invoke galactic outflows
in the Milky Way disc, while basically ALL other teams developing GCE models
for the MW do not need to include outflows in order to reproduce the disk
properties (see, e.g., Prantzos et al. 2018; Spitoni et al. 2019, 2021; Romano
et al. 2019; Kobayashi et al. 2020, Matteucci 2021 for recent work/review).
Outflows probably develop during the early formation phases, while the much
lower SFR experienced during the secular disc evolution are unlikely to favour
the development of galactic winds.
}
\doublebreak
Multi-phase kiloparsec-scale outflows are observed to be ubiquitous in Milky
Way-like galaxies (see, e.g., the review of Veilleux et al. 2020).
These outflows are indeed observed to be more metal-rich than the ISM of the
host galaxies, indicating that a portion of supernova ejecta are lost directly
to the outflow.
However, these studies find evidence of mass-loading as the metallicities are
not as high as the supernova ejecta themselves and cold-phase material is
observed in the outflows as well (Chisholm, Tremonti \& Leitherer 2018;
Cameron et al. 2021; Lopez et al. 2020, 2022).
In light of this empirical reality, we contend that it is reasonable to include
mass-loaded outflows from the Galaxy.
% As a brief procedural note, the GCE model in the present paper is adopted from
% Johnson et al. (2021),
\doublebreak
\textit{%
A different choice of the IMF slope would impact deeply the results.
Yet, this issue is surprisingly only briefly mentioned (one sentence, p.18,
l.35) and no effort is spent to investigate quantitatively the IMF issue.
}
\doublebreak
The reviewer misunderstands our motivation behind raising IMF variability as a
potential concern.
In section 4.6, we have demonstrated that the equilibrium [N/O] ratio can be
explained by the ratio of population-averaged yields of N and O, but these
quantities may be different between two environments if the IMF differs between
them.
However, in Griffith et al. (2021), we demonstrated that the yields of
different elements from massive stars depend similarly on progenitor mass.
As a result, variations in the slope of the IMF at the ~0.3 dex level produce
only 0.01 - 0.02 dex variations in the abundance ratios.
We do not consider IMF variability as a possibility in the present paper.
The review by Bastian, Covey \& Meyer (2010) finds that there is no significant
evidence for strong variations in the IMF, and although it is poorly
constrained at the low-mass end, these stars are not the dominant producers of
nucleosynthetic yields.
{\color{red} We have updated the text to clarify this point.}
\doublebreak
\textit{%
The discrepancies between model predictions and observations pointed out by the
authors when using ``off-the-shelf'' yields do not necessarily indicate that
the yields are inappropriate!
They may rather indicate that some of the underlying model assumptions are
incorrect...
\\
...
\\
Off-the-shelf stellar yields come from detailed studies that rest on stellar
evolution and nucleosynthesis THEORY and are anchored to many independent
observables.
GCE is not a theory (see Tinsley 1980) - it simply offers a framework within
which one can try to interpret the observations...
Before claiming that the stellar yields must follow some specific trends with
either stellar mass and/or metallicity basing on GCE arguments, one should be
sure that it is IMPOSSIBLE to adjust the free parameters of their GCE models to
reproduce average abundance trends with extant stellar yield grids.
}
\doublebreak
We refer the reviewer to section 4.6, where we have demonstrated that
assumptions related to the GCE model, including the strength of mass-loading,
cancel when computing the [N/O]-[O/H] relation that will arise from a given
choice of yields.
\doublebreak
\textit{%
As a matter of fact, current N yields from massive stars are found to increase
with metallicity (even a small increase as the one shown in Fig.2, left panel,
may be important when weighted with the IMF and SFH of the system...).
The finding by the authors that the yields from low- and intermediate-mass
stars by Cristallo and Ventura need an upward revision may be simply due to the
effect of the outflow, which subtracts a fraction of the newly-produced N from
the ISM. But, as already noted above, other GCE models do not need outflows at
all to explain the MW data...
}
\doublebreak
We refer the reviewer to the middle panel of Fig. 6 and the associated
discussion in section 4.2.1, where we have demonstrated that the [N/O]-[O/H]
relation can also be explained by the Cristallo and Ventura yields
without modification if the strength of mass-loading is lowered.


\end{document}
